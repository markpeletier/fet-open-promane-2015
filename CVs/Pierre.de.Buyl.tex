\begin{participant}[type=R,PM=12,gender=male,salary=5500]{Pierre de Buyl}

Postdoctoral researcher at the KU Leuven. \url{http://pdebuyl.be/}

Pierre de Buyl holds a PhD in Physics (2010) and is specialized in theoretical and
computational studies in statistical physics. He has published 15 papers in international
peer-reviewed journals and also made conference contributions (posters, articles in
conference proceedings and two invited talks). He has worked at the Université libre de
Bruxelles, the University of Toronto and is now at the KUL.
%
His early work is about the consequences of long-ranged interactions on the dynamical
evolution of models relating to plasmas and gravitation.
%
He contributed innovative insight through the direct resolution of the Vlasov equation.
This work provides an alternative to particle-based simulations and a unique point of view
on timely research questions. The resulting simulation code is available under an
open-source license and is the subject of a dedicated publication.
%
His current focus is on so-called nanomotors, a class of devices that transforms a fuel into
motion for the motor (hence the name). The operating conditions are fluctuating and strongly
out of equilibrium, in direct relation with \TheProject.

In addition to his research skills, de Buyl also participates in the organization of
scientific events (web site and database for the {\em European Conference on Complex Systems
  2012}, organizer of the conferences {\em EuroSciPy 2012} and {\em EuroSciPy 2013},
webmaster for EuroSciPy since 2013 and proceedings editor in 2013 and 2014).

de Buyl started his career as a teaching assistant and has thought for several hundred hours
already, from first year physics classes to doctoral training. He has supervised students
for research projects from the third year of Bachelor and for Master Thesis work (one at the
Université libre de Bruxelles in 2013 and one in the coming year at the KU Leuven).

\end{participant}
