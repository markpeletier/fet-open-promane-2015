\begin{sitedescription}{UNIPD} \label{desc:UNIPD}

The first ever report issued by the National Research Assessment Committee set the university of Padova as the highest ranking among leading Italian universities for the quality of its research results.
Twenty panels of evaluators, including national and international experts (25\% of whom come from foreign institutions) analysed the scientific articles, monographs, patents, works and publications presented by 77 universities, 12 public bodies and 13 private research institutes from all over Italy before awarding the University of Padova a vote of excellence in 12 out of the 20 disciplines and scientific areas covered.
In particular, the Department of Physics and Astronomy ``Galileo Galilei'' was ranked first among the physics departments of large dimension in Italy.

Based on the number of citations of articles and publications by its researchers (source: ISI), the University of Padova ranks among the top three Italian universities for total impact index, productivity index, and presence index.
Research at Padova is also attracting more and more public and private funding. Some 5.2\% of the resources that the Italian State allocates to scientific projects of national importance are awarded to Researchers at the University of Padova, while financing from the European Union for projects in various departments and research centres at the university has grown by 40\% over the last three years.
Further confirmation of the university’s ability to contribute to the cultural development and economic growth of the region comes from the number of ongoing contracts with public and private bodies for experimental activities. Commercial research services alone account for 32\% of the income of the university’s departments.

The University of Padova manages more than 200 European research projects. In 2013, 29 new projects were approved by the European Commission within the FP7. The total sum of EU contribution assigned to those projects amounts to more than 11 Million Euro.
The participation of the University of Padova in the Seventh Framework Programme has then increased: 194 is now the number of funded projects within the FP7 framework, out of which 41 managed as a coordinating body, for a total contribution of € 65 Million. Among them, 15 projects were funded by the European Research Council within the specific programme “Ideas”.
The successful participation of the University of Padova in this programme in 2012 and 2013 represents the best result among the Italian Universities. “Ideas” is the most eminent EU programme for research funding. It supports projects with a contribution that varies from 1.5 to 3.5 Million Euro. Furthermore, the University of Padova also performed in a distinguished way in other EU programmes in the last years for a total EU contribution of 3.5 Million Euro.


\subsubsection*{Curriculum vitae of the investigators}

\begin{participant}[type=PI,PM=12,gender=male]{Marco Baiesi}

  Permanent in Padova since the end of 2011, from June 2015 Associate Professor at the Department of Physics and Astronomy ``G. Galilei'' in Padova.
Marco Baiesi obtained his PhD in physics in Padova (2002), then he worked also in Leuven and Florence. Baiesi is a theoretical physicist, with an expertise in fields of polymer physics, biophysics, and nonequilibrium statistical mechanics.
 He has published 54 articles in peer-reviewed journals (including one Nature Communications and several Phys.Rev.Lett.)
He is supervising a postdoc and supervised several master students.
He is currently collaborating with colleagues in Canada, USA, Belgium, Germany, Japan, and Italy.

\end{participant}

%%% Local Variables:
%%% mode: latex
%%% TeX-master: "../proposal"
%%% End:

\begin{participant}[type=PI,PM=8,gender=male]{Attilio L. Stella}

Full professor of Theoretical Physics at the University
of Padova. Attilio Stella's scientific activity is in the field of statistical mechanics,
both equilibrium and nonequilibrium, with particular recent interest in biologically 
inspired problems, polymer statistics and econophysics. He has been PI 
of a number of national research projects and has a good record of participation
in international conferences as invited speaker or as organizer. He is author of about
150 articles in peer-reviewed journals and has been supervisor of 15 PhD theses.

\end{participant}

%%% Local Variables:
%%% mode: latex
%%% TeX-master: "../proposal"
%%% End:


\begin{participant}[type=res,PM=36,salary=2140]{NN}
Young postdoc with a strong background in statistical mechanics and numerics.
\end{participant}

\begin{participant}[type=res,PM=36,salary=2530]{NN}
Experienced postdoc with a strong background in statistical mechanics, numerics, and material science.
\end{participant}

\subsubsection*{Publications, achievements}

\begin{compactenum}
\item Coauthoring many papers in the field of statistical mechanics and complexity, in journals as Proc. Nat. Acad. Sci, Nature Communications, and Phys. Rev. Lett.
\end{compactenum}


\subsubsection*{Previous projects or activities}

\begin{compactenum}
\item Current organization of the next international school ``Fundamental problems in Statistical
Physics'', in Brixen (2017).
\end{compactenum}

\subsubsection*{Significant infrastructure}

The offices will be in the Department od Physics and Astronomy of Padova.

\end{sitedescription}



\begin{draft}
\vspace{1cm}\TOWRITE{PAR1P1}{Complete check list below -- delete completed items if you wish}

\begin{verbatim}
- [ ] checked that sum of person months put into finance request is
  the same as sum of person months associated with the Work Packages
  (in proposal.tex, as defined as part of the \begin{workpackage}"
  command.
  
- [ ] completed site specific resource summary in resources.tex,
  including table of non-staff costs.

\end{verbatim}
\end{draft}

%%% Local Variables: 
%%% mode: latex
%%% TeX-master: "../proposal"
%%% End: 
