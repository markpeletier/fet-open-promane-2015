\providecommand{\classoptions}{keys}
\documentclass[noworkareas,deliverables,\classoptions]{euproposal}       % for writing
%\documentclass[submit,noworkareas,deliverables]{euproposal}        % for submission
%\documentclass[submit,public,noworkareas,deliverables]{euproposal} % for public version

\usepackage[utf8]{inputenc}

\usepackage{float}  % used to suppress floating of tables in Resources section.
\usetikzlibrary{calc,fit,positioning,shapes,arrows,snakes}

\addbibresource{bibliography.bib}
%%% institutions
\WAinstitution[id=KUL,
        countryshort=BE,
        acronym=KUL]
        {KU Leuven}

\WAinstitution[id=Padova,
        countryshort=IT,
        acronym=Padova]
        {University of Padova}

\WAinstitution[id=Leipzig,
        countryshort=DE,
        acronym=Leipzig]
        {Universität Leipzig}

\WAinstitution[id=USTUTT,
        countryshort=DE,
        acronym=USTUTT]
        {Universität Stuttgart}

\WAinstitution[id=TUE,
        countryshort=NL,
        acronym=TUE]
        {Technische Universiteit Eindhoven}

\WAinstitution[id=FZU,
        countryshort=CZ,
        acronym=FZU]
        {Institute of Physics of the Czech Academy of Sciences}

\WAperson[id=cmaes,
           personaltitle=Prof.,
           birthdate=1 Jan. 2000,
           academictitle=Professor,
           affiliation=KUL,
           department=Instituut voor Theoretische Fysica,
           privaddress=None of your business,
           privtel=that neither,
           email=christian.maes@fys.kuleuven.be,
           workaddress={Celestijnenlaan 200D, B-3001 Leuven, Belgium},
           worktel=+32 16 32 72 33,
           workfax=N/A
           ]
           {Christian Maes}

%%% Local Variables: 
%%% mode: latex
%%% TeX-master: "proposal"
%%% End: 
 % Some sections of the included files depend on this.
\usepackage{comments}
\usepackage{framed}

\defbibenvironment{proposal-env}
{\inparaenum%
\renewcommand*\labelenumi{[\theenumi]}}
{\endinparaenum}
{ \item}

\setdefaultleftmargin{1em}{0.5em}{0.5em}{0.5em}{0.5em}{0.5em}

\defbibheading{proposal-bib}[References:]{\noindent {\bf #1}}

\begin{document}

\begin{proposal}[
  % These PM numbers (person months) are for the coordinator to help planning
  % Participants should not change these, but add PM numbers in the CVS in
  % the site descriptions at CVs/*.tex
  site=KUL,
  site=TUE,
  site=ULEI,
  site=UNIPD,
  site=FZU,
  site=USTUTT,
  botupPM, % we want to work via bottom up PM distribution,
  coordinator=cmaes,
  coordinatorsite=KUL,
  acronym={PROMANE},
  acrolong={PROMANE},
  title=Probing Macroscopic Nonequilibria,
  callname=FET-OPEN Research and Innovation actions 2014-2015,
  callid=H2020-FETOPEN-2014-2015-RIA,
  keywords={nonequilibrium thermodynamics, new materials, transport and control theory},
  instrument=Horizon 2020 - Future and Emerging Technologies,
  months=48,
  compactht]
\newcommand{\TheProject}{\pn}% \pn is defined automatically

\begin{abstract}

\TheProject will develop a control methodology within nonequilibrium statistical
thermodynamics, promoting irreversible thermodynamics and dynamical systems to a unified tool for
industrial
applications which includes the recent progress in fluctuation and response
theory.
New technologies will benefit from the manipulations of and the control via macroscopic
nonequilibria. These are nonlinear systems that are are subject to nonconservative forces or
driven from contacts with multiple reservoirs that impose conflicting thermodynamic
behavior, or for which the constituents are active, {\it i.e.}, objects that could be
self-propelling. Our main hypothesis is that macroscopic nonequilibria can be used for
steering immersed subsystems or probes, thus developing a control theory using nonequilibria
together with the possibility of creating new macroscopic phases.
%
The expected deliverables include enhanced stabilization of metastable phases and the
emergence of new response and transport behavior, crucial items in developing systematically
interesting devices and material properties.
For technological avenues, our work will reveal important insights, experimental and
computational methods for example by providing far-from-equilibrium relaxation times of
viscoelastic media, for targeted catalysis, for clustering control or for viscosity
reduction in industrial processing. Other possible applications involve noninvasive
surgeries and drug delivery in medicine, or the design of smart materials with built-in
actuation mechanisms.
%
Previous efforts focused on the destructive and information-erasure roles of dissipation.
Time has come to comprehend and exploit the
constructive role and possibilities of nonequilibrium conditions. Complementary to entropy
production, changes in kinetics and dynamical activity make statistical forces nongradient
and nonadditive, allowing unseen and novel behavior. Controlling it opens a new paradigm and major
opportunities for future technology.

\end{abstract}

\newpage

\ifsubmit\else\setcounter{tocdepth}{4}\fi
\tableofcontents


% ---------------------------------------------------------------------------
%  Section 1: Excellence
% ---------------------------------------------------------------------------

\section{S\&T Excellence}
\subsection{Targeted breakthrough, Long term vision and Objectives}\label{sec:objectives}

\eucommentary{
  \begin{compactitem}
  \item Describe the targeted scientific breakthrough of the project.
  \item Describe how the targeted breakthrough of the project contributes to a
    long-term vision for new technologies.
  \item Describe the specific objectives for the project, which should be clear,
    measurable, realistic and achievable within the duration of the project.
  \end{compactitem}
}

\paragraph{Scientific and technological goals}

The project is aimed at exploiting and controlling the motion of probes and devices in and
through contact with nonequilibrium media. ``Nonequilibrium media'' refers to an environment
(solvent, bulk of a material, etc) that is active or that is driven away from its
thermodynamic equilibrium.
%
The {\bf scientific
  breakthrough} involves an operational formulation of nonequilibrium statistical mechanics.
That goal implies the transition to a third major stage of {\bf fundamental nonequilibrium
  research}. Most often in the past, the focus has been either on studying the approach to equilibrium or on
the derivation and description of dissipative transport of mass, energy and momentum between
various equilibrium reservoirs.  By deriving and using the precise relation between
systematic forces, friction, and noise on probes, we open a new and realistic class of
dynamical systems for which the control lies in the nonequilibrium environment.
Complementarily, the understanding of the fluctuation and response behaviour of nonequilibrium
media allows us to extend the standard irreversible thermodynamic theory as it was conceived in
1930--1970.

Both, new control of emerging dynamical systems as probes connected to nonequilibria and the manipulation of kinetic to thermodynamic behaviour of media are essential
ingredients in future technology and industrial applications.  To cite a few general
far-reaching objectives: the controlled motion or transport of material through living
tissue or in interaction with life processes requires fundamental departures from
traditional transport theory; the stabilization of coherence as needed for future quantum
technology cannot be obtained within thermal environments but could be enhanced via
nonequilibrium contacts; mobilities and conductivities close-to-equilibrium are subject to
the fluctuation-dissipation relation but nonequilibrium driving can produce unseen transport
and conductivity behavior for the realization of new materials. 


 As is central to the goals
of the FET-open initiative, theoretical and experimental foundational work will lead to
create abilities and conditions under which technological innovation becomes possible.  For
example, serious conceptual problems need to be solved before active-particle suspension
will become a common basis for enhanced targeted catalysis or for fluid mixing and viscosity
reduction in industrial processing, or for noninvasive surgeries and drug delivery in
medicine, or for the design of smart materials with built-in actuation mechanisms. These
open conceptual issues are addressed via several pathways in the theoretical parts of the
proposal.  Leaping into macroscopic nonequilibria moreover implies that classical
measurement techniques, developed for equilibrium systems, will generally fail to work as
usual and will have to be replaced by new designs. Brownian thermometry provides an example
that is relevant to a variety of applications \cite{kroy:2014}, and which will be
generalized to conditions far from equilibrium, in the project.  On the practical and
applied side, radically new manipulation techniques are required to fully exploit the
innovative potential provided by active particles. As pertinent examples, we mention the
thermophoretic trap and the photon-nudging method pioneered by the experimental group in
Leipzig \cite{Qian2013,Braun:NanoLetters:2015}.  Both techniques steer particles (passive
and active particles in the trap and hot active particles in the nudging technique,
respectively) by exploiting far-from equilibrium conditions in the solvent. Notably, they
work without imposing external forces, via directly addressing the nonequilibrium solvent
conditions and thereby the (self-)thermophoretic propulsion (or ''swimming'') of the
particles. Their usefulness is enhanced by employing Maxwell-demon type control
strategies. Prototypes of these genuinely nonequilibrium techniques exist and will be
employed and refined (e.g. by optimizing dedicated control strategies) within the project.
One should realize that the assumption of a thermally equilibrated bath is not fulfilled in
most industrially (oil, polymer melts, dense colloidal or micellar suspensions) or
biologically relevant (blood, mucus, DNA-solutions) liquids which display visco-elastic
properties, i.e. elastic behaviour like a solid and viscous behavior like a fluid. As a
consequence, such fluids exhibit large relaxation times (up to seconds), and can thus easily
be driven out of thermal equilibrium by a forced colloidal probe. Within this proposal, we
want to study how the local perturbation of visco-elastic systems affects the dynamics and
effective interactions of colloidal particles in such media with particular view on
e.g. memory effects, many body interactions and local non-linear rheological properties.
Similarly, time-dependent phenomena at very long and even at very fast time scales, such as
in glasses or electronic devices, elude the usual equilibrium thermodynamic
approach.  Such studies will in general reveal the properties of nonequilibrium baths which
are of manifold importance.

More generally and as part of the {\bf long term vision} we believe that a systematic
understanding of nonequilibrium response and fluctuation behavior, beyond the traditional
linear response regime, will uncover new parameters to control technological processes, to
steer motion, and to explore and stabilize new interesting material properties.

\paragraph{Objectives}

can be divided into two major classes:\newline
\begin{inparaenum}[A.]
\item identifying the key-control and manipulation parameters for probes and devices in contact with nonequilibrium media.  That means to understand the emerging dynamical system immersed in nonequilibria.  It will lead to improved transport quality in environments such as the turbulent atmosphere, in active media or in nonlinear fluids.  Besides transport, we expect also the stabilization and the emergence of interesting device properties with long time coherence and classical dia- and paramagnetic phases as a couple of important examples;\newline
\item shaping nonequilibrium media for their response behavior.  The addition of kinetic and nonequilibrium parameters in material design can lead to absolutely new effects in response or susceptibility. Negative compressibilities, thermal or variable conductivities introduce new possibilities and applications for flexible and smart materials.  We provide a new computational framework for {\it ab initio} calculations of nonequilibrium phase diagrams, opening an important tool in material and device production.
\end{inparaenum}


We will provide modelling and simulation techniques for the simulation of probes and
devices in contact with nonequilibria.  These are high level molecular dynamics codes for
simulating and calculating the emergent behavior of contacts and probes.  These techniques
will be made available through all  work packages. It is an important objective
to make that tool also ready to future industrial partners  who are challenged by the complexity of
dealing with nonequilibria.


We will offer a scientific basis and new avenues for the realization of new properties of
matter (in response and transport), and the enhanced stability of thermodynamic phases or
behaviour.  See further also in \WPref{WPcompress} for the conception of
meta-materials. Specifically we have good understanding for workable implementations of
negative compressibilities.


For the control of colloidal motion in nonequilibrium baths and in nonlinear media,  \WPref{WPbrownian}, we use theoretical and experimental studies of probes in visco-elastic
solvents.  Most fluids found in industry and biology are visco-elastic and/or out of
equilibrium.  Control of shear, the development of new separation techniques based on
dynamical activities and steering of clustering properties are achievable in the current
project.


Many challenges are kinetic rather
than thermodynamic, especially for processes and manipulations in nonequilibrium
environments or for time-dependent parameters, \WPref{WPactive}.  Irreversible thermodynamics cannot deal with
the overwhelming nature of kinetic control, but nonequilibrium statistical insights point to
the control of dynamical activity; what some of us have called the frenetic contribution.
One objective is to make a frenometer for operational control and measurement of dynamical
activity.  That is essential for the control and steering of dynamics out-of equilibrium.
For example, polymer physics outside equilibrium will supply a necessary and complementary
component to the vast domain of polymer research with thermodynamic control.


We develop the theory of statistical forces on probes in contact with macroscopic
nonequilibria, essential for controlling motion in turbulent media (atmosphere dynamics) or
under biological flow (blood streams).  \WPref{WPdissipation} and \WPref{WPcore} will work towards experimental
and observational predictions of new emergent behavior for probes in nonequilibria, as
resulting from the essentially nongradient, nonadditive and nonlocal features of these
nonequilibrium statistical forces.


These objectives find further realizations in our existing contacts with colloidal and
polymer scientists, micro-electronic and microbio-engineers and atmosphere researchers. {\bf The overarching goal
remains to make available the science of nonequilibria to technological and societal
developments.}



\subsection{Relation to the work programme}\label{sec:relation-wp}

\begin{compactdesc}
\item[Long-term vision] \TheProject enables a change of paradigm across many technological
fields on the basis of nonequilibrium thermodynamics. Current technologies rely on an
understanding of thermodynamics that dates from half a century ago and we will deliver the
set of tools to scientists and to the industry that is missing.
\item[Breakthrough S\&T target] The developments within \TheProject rely directly on
technological applications, either via relevant physical models or via the experimental
partners (\site{USTUTT} and \site{ULEI}). Advanced or novel nonequilibrium theories are
currently not used in the industry and we will provide the building blocks to enable them
for the development of new material (\WPref{WPcompress}), ultra fast processes (see XXX)
and control strategies (\WPref{WPdissipation}) and viscoelastic media (\WPref{WPbrown}).
\item[Novelty] The changes that we are proposing do not aim at a minor modification of the
existing processes but at replacing them with a better understanding and, more specifically,
new strategies for diagnosis and control.
\item[Foundational] Our work will have a decisive impact on other fields in which the
control of devices lacks a comprehensive thermodynamical understanding: medicine and
pharmacology (control of nanoparticles in living systems and targeted delivery mechanisms),
engineering (fast industrial processes, see XXX).
\item[High risk] Scientific research carries risks by its very nature, as we are pushing the
frontiers of knowledge. It may be the case that our ideas stumble on mathematical or
experimental challenges that would require efforts beyond the 4 yeards of the
project. Still, the combined expertise in \TheProject offers strong chances of success and
even before completion many exciting outputs are foreseen (see XXX).
\item[Interdisciplinarity] \TheProject consists in two experimental physics groups
(\site{ULEI}, \site{USTUTT}), one mathematical group (\site{TUE}), two theoretical physics
groups (\site{KUL}, \site{UNIPD}), one of which specialized in simulation methods
(\site{UNIPD}). This combination allows a consistent organization of the work across work
packages to deliver results ranging from the conceptual to the proof of principle.
\end{compactdesc}

\subsection{Novelty, level of ambition and foundational character}\label{sec:progress}

\eucommentary{
  \begin{compactitem}
  \item Describe the advance your proposal would provide beyond the
    state-of-the-art, and to what extent the proposed work is ambitious, novel
    and of a foundational nature. Your answer could refer to the ground-breaking
    nature of the objectives, concepts involved, issues and problems to be
    addressed, and approaches and methods to be used.
  \end{compactitem}
}

The first stage of modern thermodynamics consisted in microscopic studies of equilibrium systems, giving rise to equilibrium statistical mechanics. The current body of knowledge concerning nonequilibrium
physics represents, as we have outlined in Sec.~\ref{sec:objectives}, the {\em second stage}
of the domain: the study of transport and response theory close to
equilibrium.

\paragraph{Non dissipative nonequilibrium physics}
Since about two decades, nonequilibrium studies have revisited steady regimes further away
from equilibrium, and a fluctuation-response theory is  emerging.
The recent idea of dynamical activity starts a new line of conceptual
understanding for today's nonequilibrium thermodynamics. The focus on
entropy production has brought many insights in the last 150 years (in physics and other
fields as well, such as information theory) but is however restrictive.  We emphasize the role of excess thermodynamic quantities in relaxation to {\it non}equilibria, and we find that a crucial role is also being played by time-symmetric fluctuations.
Volatilities or time-symmetric traffic, frenesy and other names have been given to this dynamical activity which is all important  for a valid fluctuation-response theory away from equilibrium.
 Our current
understanding of time-symmetric quantities remains rather formal and most results concern
specific systems (often ``toy models'') , be it of stochastic or highly chaotic nature.
%
\TheProject chooses specific questions, some mathematically very challenging and some very
pragmatic, in fluctuation-response theory to open the meaning of dynamical activity, and to
make it operational.
%
Explicitly, we focus on consequences that can be observed and measured, and on the
experimental control that this theory brings.

\paragraph{Probing nonequilibrium}

We implement the control and monitoring of a system from its contact with a nonequilibrium
medium.
%
The newly found properties allow the design of devices or subsystems showing unusual
but very much wanted features like increased stability or coherence, negative differential
response on command and self-organization like flocking, clustering or pattern formation.
%
That is not to be done by explicitly producing the required feedback dynamical system, but
rather by monitoring kinetic and thermodynamic aspects of nonequilibrium baths.
%
A first question is about the behaviour of probes in active contact with a nonequilibrium
sea; what are the induced systematic statistical forces?
%
In equilibrium (that is, when the probe is in contact with an equilibrium reservoir) the
resulting statistical forces can be derived from a gradient, reminding what is done in
Newtonian mechanics.
%
The result there is that it allows us to draw free energy landscapes to understand the
changes in the system, as if it concerned a conservative mechanical system for which we give
the potential energy.
%
Moreover, always for equilibrium, the statistical forces are additive. For instance, bulk
contributions can easily be separated from boundary contributions. The very possibility of
thermodynamic behaviour, like the presence of an equation of state, depends on that
distinction.
%
All of that need not
and in general, will not, be true for probes (systems, walls, collective coordinates) in
contact with (genuine) nonequilibrium media or reservoirs. That has interesting
consequences in the manipulation of such forces, adding oscillatory components, going from
attractive to repulsive and obtaining nonvanishing resulting forces that otherwise, in
equilibrium, would be zero from mutually canceling contributions.  
%
It also means that the stationary positions of the probe can be at different locations
compared with equilibrium, possibly increasing the stability of phases that would otherwise
(in equilibrium again) be unstable.

The Kapitza oscillator is a pendulum that remains stable upright when being shaken. This
example is a rather simple dynamical system and we aim in \TheProject at a systematic
extension of this idea for the stability of macroscopic phases that are very unlikely in
equilibrium.
%
Needless to add here that this may have dramatic consequences on the
phase diagram for systems in contact with nonequilibrium media, not only breaking the Gibbs
phase rule but also introducing new phases of matter.  That is certainly one of the most
exciting possibilities of explorations in the present project. How are for example effects
like homeostasis and adaptation linked to the nonequilibrium features of the medium and its
contact with the device?  A related consequence of contact with nonequilibrium reservoirs is
the possibility of population inversion, which can imply that the effective temperature of
the probe for example is much larger.


\subsection{Research methods}\label{sec:methods}
\eucommentary{
  \begin{compactitem}
  \item Describe the overall research approach, the methodology and explain its
    relevance to the objectives.
  \item Where relevant, describe how sex and/or gender analysis is taken into
    account in the project's content.
  \end{compactitem}
}

The fundamentally connected components in this \TheProject have mathematical, conceptual,
computational and experimental sides. The dependent character of these parts is outlined in
the chronology of the WPs in Sec.~\ref{sec:wp}.

\begin{asparaenum}
\item {\bf Mathematical methods:} We use stochastic calculus for describing the model
systems, including their spatially extended and many-body characters.
%
\site{TUE} and \site{KUL} are recognized experts on applied mathematics and
mathematical physics.
\item {\bf Theoretical and conceptual framework:} The proper understanding of thermodynamics
relies in a consistent definition of the entropy fluxes and of the dynamical activity.
%
We will develop an operationally oriented approach to nonequilibrium problems.
%
All nodes will participate in this effort.
\item {\bf Computational support:} Numerical simulations will serve as practical
implementation of the theoretical model systems and as a supporting tool for the
experiments.
%
They allow us to obtain the most complete description of the physical processes and test our ideas beyond what is analytically possible.
%
The computational developments are led by \site{KUL} and \site{UNIPD}.
%
For the more realistic scenaries (i.e. for modeling the experiments), Molecular Dynamics
simulations will be used.
\item {\bf Experimental work:} The soft condensed and liquid matter laboratories
(\site{ULEI}, \site{USTUTT}) are familiar with the techniques needed for active materials
and particles.
%
They are experts on the optical control and analysis of colloidal system and on the physical interpretations of the outcomes.
\end{asparaenum}

\subsection{Interdisciplinary nature}\label{sec:interdisc}

\eucommentary{
  \begin{compactitem}
  \item Describe the research disciplines involved and the added value of the inter-disciplinarity.
  \end{compactitem}
}

The consortium joins experts in applied mathematics, mathematical physics, statistical mechanics, polymer science, computational physics, liquid and soft condensed matter physics and fluid mechanics.
The general theme of nonequilibrium is indeed very  large, and amazing examples of nonequilibria span a great variety of phenomena ranging from nano-technology, over biology to atmosphere physics.
The researchers involved in the present project have in their own work contributed to the understanding of these various nonequilibrium sides.  Leipzig and Padova nodes have many contributions in biophysics, while Stuttgart is famous for colloidal physics, liquid matter experiments and the study of Casimir forces.  Leuven and Eindhoven nodes represent the more mathematical corners, with emphasis on analytical and stochastic methods.  Still other expertise (like in Padova and Leuven) is present in solving computational and simulation problems.   At the same time, the various labs have closer contacts, both scientific and more industrial, in a variety of disciplines, like sensor-technology, water transfer in porous media, polymer chemistry, financial mathematics, weather prediction and food science to mention a few.  Also the relation with students and young researchers spans different faculties, with science projects in engineering faculties, in institutes for cultural studies, and of course in computer science, chemistry and physics departments.



%%% Local Variables:
%%% mode: latex
%%% TeX-master: "proposal"
%%% End:


% ---------------------------------------------------------------------------
%  Section 2: Impact
% ---------------------------------------------------------------------------

\section{Impact}
\subsection{Expected impacts}

\eucommentary{
  Please be specific, and provide only information that applies to the proposal
  and its objectives. Wherever possible, use quantified indicators and targets.
%
  \begin{compactitem}
  \item Describe how your project will contribute to the expected impacts set out
    in the work programme under the relevant topic.
  \item Describe the importance of the technological outcome with regards to its
    transformational impact on science, technology and/or society.
  \item Describe the empowerment of new and high-potential actors towards future
    technological leadership.
  \end{compactitem}
}

\TheProject is based on operational objectives and will be implemented experimentally and
via numerical simulations.
%
The nature of the tools that we develop is however adequate to serve for many more
applications, as the last century of statistical mechanics developments has shown.
%
Nonequilibrium theory and phenomenology reach even much richer grounds and connects us to
\begin{compactitem}
\item Biological systems and medical applications, in which noninvasive diagnosis and
micro-manipulation of macromolecules XXX.
\item Design of new materials and monitoring of material properties beyond our ideas on
compressibility. Indeed, the response of a material can also be of thermal or chemical
nature.
\item Cybernetics of active media and the steering of active particles in biological
(living) environments is expected to revolutionize medical interventions and pharmacy via
the targeted transport of drugs and new microinvasive treatments and therapies.
\item Micro-bioreactors are small chemical factories whose direct control is
unlikely. Thermodynamical control on the other hand is a robust tool.
\item Quantum devices require the control of many body coherence and stability for reaching
macroscopic phases that are only metastable under equilibrium conditions. This is among
other fields relevant for quantum computing applications.
\end{compactitem}

We see in all of the above items the ubiquitous role of the nonequilibrium paradigm. Its
transformational impact on science, technology and society are enormous and largely
unexploited. \TheProject promises a serious start toward these applications via
nonequilibrium physics.
%
Nonequilibrium theories are currently not used in the industry and we will provide the
building blocks to enable it. We expect that the developement of nonequilibrium theories
will deliver a competitive advantage, similar in principle to the one delivered by James
Watt developments for the steam engine.

\subsection{Measures to maximise impact}

\paragraph{a) Dissemination and exploitation of results}

The major method of dissemination of our results is first the standard scientific practice
of open access publishing, and participation in international conferences and discussion
fora. The universities have important research and development centres where starting
spin-offs find resources and support.
%
There will be important contacts with these centres.

Dissemination of the results is done by the free web archive, in publications in the
standard specialized scientific journals (mostly non-commercial), by talks, by schooling and
in contacts in conferences. As members of the Editorial Boards of Journal of Physics A  (IOP), of the Journal of Statistical Physics (Springer), of Annales Henri Poincar\'e, of
Fundamental Theories of Physics (Springer) and of the liquid matter board of the European Physical Society, etc we can organize special issues and create
special volumes dedicated to aspects of the project. Indeed, as the project deals with some
very novel aspects that have not appeared in standard or even not so standard treatments of
irreversible thermodynamics, an important effort will be necessary and will be made to reach
also less specialized researchers as well as scientists involved in possible applications.
%
Obviously an important cost then also goes to traveling and participation in conferences and
international discussions. Again, since the project is not entirely mainstream and some
concepts/questions are quite new, the project needs to invest in visibility and in numerous
exchanges on international platforms.  All team members will be
expected to show great mobility.  During the years, the various members meet once at each
node with everybody, at a rate of roughly one meeting every 10 months. PIs stay a few days
for talks of everyone and discussions. Postdocs and students can stay
longer, extending their visits for a month. That will allow perfect exchange of ideas and
progress.  Similar considerations apply for inviting scientists where
costs include traveling, housing and subsistence.  The project will also invest (still under other
direct costs) in bringing together various points of view on the role of dynamical activity
and the construction of nonequilibrium statistical mechanics. For that reason we foresee to
organize two major conferences, one at around month 20, the other towards the end of the
project.  


\eucommentary{
  \begin{compactitem}
  \item Provide a plan for disseminating and exploiting the project results. The
    plan, which should be proportionate to the scale of the project, should
    contain measures to be implemented both during and after the project.
  \item Explain how the proposed measures will help to achieve the expected
    impact of the project.
  \item Where relevant, include information on how the participants will manage
    the research data generated and/or collected during the project, in
    particular addressing the following issues\footnote{For further guidance on
      research data management, please refer to the H2020 Online Manual on the
      Participant Portal.}:
    \begin{compactitem}
    \item What types of data will the project generate/collect?  o What
      standards will be used?
    \item How will this data be exploited and/or shared/made accessible for
      verification and re-use?  If data cannot be made available, explain why.
    \item How will this data be curated and preserved?
    \end{compactitem}
 %      
    You will need an appropriate consortium agreement to manage (among other
    things) the ownership and access to key knowledge (IPR, data etc.). Where
    relevant, these will allow you, collectively and individually, to pursue
    market opportunities arising from the project's results.\\
%
    The appropriate structure of the consortium to support exploitation is
    addressed in section~3.3.
%
  \item Outline the strategy for knowledge management and protection. Include
    measures to provide open access (free on-line access, such as the 'green' or
    'gold' model) to peer- reviewed scientific publications which might result
    from the project.%
    \footnote{Open access must be granted to all scientific publications
      resulting from Horizon 2020 actions. Further guidance on open access is
      available in the H2020 Online Manual on the Participant Portal.}\\
%
    Open access publishing (also called 'gold' open access) means that an
    article is immediately provided in open access mode by the scientific
    publisher. The associated costs are usually shifted away from readers, and
    instead (for example) to the university or research institute to which the
    researcher is affiliated, or to the funding agency supporting the research.\\
%
    Self-archiving (also called 'green' open access) means that the published
    article or the final peer-reviewed manuscript is archived by the researcher
    - or a representative - in an online repository before, after or alongside
    its publication.  Access to this article is often - but not necessarily -
    delayed ('embargo period'), as some scientific publishers may wish to recoup
    their investment by selling subscriptions and charging pay-per-download/view
    fees during an exclusivity period.
%
  \end{compactitem}
}

\paragraph{b) Communication activities}

\eucommentary{
  \begin{compactitem}
  \item Describe the proposed communication measures for promoting the project
    and its findings during the period of the grant. Measures should be
    proportionate to the scale of the project, with clear objectives. They
    should be tailored to the needs of various audiences, including groups
    beyond the project's own community. Where relevant, include measures for
    public/societal engagement on issues related to the project.
  \end{compactitem}
}

Various members of the team have larger networks of communication both towards the general
public (public outreach) as towards industrial players and groups of interest.  The first
aim there is to transfer knowledge and tools, and to help in optimizing public interest. An
important initiative will be the organization of ``physics meets industry days'' where each
time during a week, specific problems of industry or economic activity will be presented.
They will be treated by students and experts towards helping to solve these problems, with
direct feedback toward the industry of company.  Such initiatives exist already in some
countries but will be started up in other European countries, and with an additional
selection and expertise platform related to complex and nonequilibrium phenomena.



%%% Local Variables:
%%% mode: latex
%%% TeX-master: "proposal"
%%% End:


% ---------------------------------------------------------------------------
%  Section 3: Implementation
% ---------------------------------------------------------------------------

\section{Implementation}

\input{implementation}

\gantttaskchart[draft,xscale=.33,yscale=.33,milestones]

\subsubsection{Deliverables}\label{sec:deliverables}
\inputdelivs{9.3cm}

\subsubsection{Milestones}\label{sec:milestones}
\eucommentary{Milestones means control points in the project that help to chart progress. Milestones may
correspond to the completion of a key deliverable, allowing the next phase of the work to begin.
They may also be needed at intermediary points so that, if problems have arisen, corrective
measures can be taken. A milestone may be a critical decision point in the project where, for
example, the consortium must decide which of several technologies to adopt for further
development.}

\begin{milestones}
\milestone[id=start, month=1, verif={Public announcement of open positions.}]
%
{Starting up}
%
{Formal start of the project: Open the research positions, set up the website.}

\milestone[id=models,month=12, verif={}]
%
{Systems setup}
%
{Set up the experiments and the simulation software.}

\milestone[id=framework,month=24, verif={Availability of theoretical models. Publication of
  first technological report.}]
%
{Common theoretical framework}
%
{The physical properties of all model systems, theoretical and experimental, are formulated
  in a unified manner. The role of nonequilibrium in the properties of the systems is given
  a meaning.}

\milestone[id=data1,month=24, verif={Availability of experimental data.}]
%
{Experimental results}
%
{The first round of experiments provides validated data.}

\milestone[id=final,month=48, verif={Publication of final report and second
  technological report.}]%
{Final milestone}
%
{Join the theoretical model and experimental results in joint publications}

\end{milestones}

%%% Local Variables:
%%% mode: latex
%%% TeX-master: "proposal"
%%% End:



% ---------------------------------------------------------------------------
% Include Work package descriptions
% ---------------------------------------------------------------------------

\newpage
\subsubsection{Work Package Descriptions}\label{sec:workpackages}
%% WP titles and order are defined in deliverables.tex
%%% work package style may be broken -- fix this!!

%% Local WP number counter - should possibly be global and hidden?
\begin{workplan}
\begin{workpackage}[id=management,type=MGT,wphases=0-48!.2,
  title=Project Management,short=Management,
  lead=KUL,KULRM=24]

\begin{wpobjectives}

Provide efficient communication and practical support for the project's events,
including IT support.
%
Serve as contact for the media.
%
Ensure timely and consistent reporting on the project advance and monitor the
progress of other work packages and deliverables.
%
Coordinate the report on technological perspectives.

\end{wpobjectives}

\begin{wpdescription}

The coordinator node, via the project manager, oversees the project from the
administrative point of view. Each node remains responsible for its internal
(accounting, hiring, etc) management while general project issues will be
handled at the KUL, including communication to and from the European Commission.

\end{wpdescription}

\begin{tasklist}
\begin{task}[title=Reporting,id=mgt-reporting,lead=KUL,PM=6,wphases=0-48!0.125]

Preparation of yearly reports, including the project's final report.

\end{task}

\begin{task}[title=Meetings,id=mgt-meetings,lead=KUL,PM=6,wphases=0-48!0.125]

Organization of the kickoff and subsequent meetings.

\end{task}

\begin{task}[title=IT,id=mgt-IT,lead=KUL,PM=6,wphases=0-48!0.125]

Setup and regular update of the project's website.
%
Preparation of data, when needed, for sharing via open data repositories.

\end{task}

\begin{task}[title=comm,id=mgt-comm,lead=KUL,PM=6,wphases=0-48!0.125]

Preparation of media content for the communication of the project: press
releases, videos, illustrations.

\end{task}
\end{tasklist}

\begin{wpdelivs}
  \begin{wpdeliv}[due=1,id=ca,dissem=CO,nature=R,lead=KUL,miles=start]{Consortium Agreement}
  \end{wpdeliv}
  \begin{wpdeliv}[due=1,id=website,dissem=PU,nature=DEC,lead=KUL,miles=start]{Web site}
  \end{wpdeliv}
  \begin{wpdeliv}[due=6,id=hosting,dissem=PU,nature=R,lead=KUL]{Hosting}
  \end{wpdeliv}
  \begin{wpdeliv}[due=24,id=media,dissem=PU,nature=R,lead=KUL]{General public media}
  \end{wpdeliv}
  \begin{wpdeliv}[due=24,id=TR1,dissem=PU,nature=R,lead=KUL,miles=framework]{Technological Report 1}
  \end{wpdeliv}
  \begin{wpdeliv}[due=48,id=finalreport,dissem=CO,nature=R,lead=KUL,miles=final]{Final report of the project}
  \end{wpdeliv}
  \begin{wpdeliv}[due=48,id=TR2,dissem=PU,nature=R,lead=KUL,miles=final]{Technological Report 2}
  \end{wpdeliv}
\end{wpdelivs}

\end{workpackage}

%%% Local Variables: 
%%% mode: latex
%%% TeX-master: "../proposal"
%%% End: 

\begin{workpackage}[id=WPactive,wphases=0-48,
  short=Active Particle Suspensions,% for Figure 5.
  title=Probing Active Particle Suspensions with Colloids and Polymers,
  lead=Leipzig,
  LeipzigRM=96,PadovaRM=6,USTUTTRM=2]

\newrefsection

\begin{wpobjectives}
The objectives of this WP are:
  \begin{compactitem}
  \item \textbf{Non-Equilibrium Equations of State (NEOS):} explore the existence of
  macroscopic non-equilibrium equations of state for active Brownian particle suspensions
  with microscopic many-particle models
  \item \textbf{Active Crowding:} elucidate by analytical model calculations how mechanical
  and chemical activity (such as by molecular motors in the cytoplasm of animal cells)
  affects the statistical mechanics of embedded biopolymers (such as protein fibers or DNA)
  \item \textbf{Noise Temperature:} validate the concept of a frequency-dependent noise
  temperature in non-equilibrium fluctuation-dissipation relations and the failure of
  Onsager's regression hypothesis for colloidal probes in non-isothermal solvents by
  experiments and computer simulations
  \item \textbf{Active-Particle Suspensions:} scrutinize tentative non-equilibrium equations
  of state by a combination of experiments and all-atom non-equilibrium molecular-dynamics
  simulations
  \end{compactitem}
\end{wpobjectives}

\begin{wpdescription}
In equilibrium, information about the macroscopic state of a medium can be deduced via
rather simple measurement procedures using simple probes and then be summarized in an
equation of state.
%
In the future, something similar may become possible even far from equilibrium, and
revolutionize our technological abilities, but this will require much more sophisticated
probing and analyzing procedures that are not yet generally understood.
%
Far from equilibrium, fluctuations are non-universal and standard thermodynamic notions,
such as temperature and pressure, become ambiguous.
%
This has many unintuitive practical consequences, which defy conventional thermodynamic
wisdom and potentially invalidate and doom to failure well-established standard
technological procedures.

For example, a colloidal particle in a non-isothermal solvent cannot be used for Brownian thermometry 
in the conventional way \cite{rings-etal:2010,kroy:2014}, and also the proper theoretical 
treatment of its systematic, thermophoretic propulsion is the subject of ongoing debates.
%
We have recently shown how to overcome some of these limitations in terms of a frequency-dependent noise 
temperature \cite{falasco-etal:2014}.
%
We want to validate this result experimentally and to see to what extent something similar be achieved for other 
non-equilibria, particularly in active-particle suspensions.

Further, far from equilibrium, the mechanical pressure on a mesoscopic probe will generally hinge on the range and 
symmetry of the medium-probe interactions: if an active-particle suspension is compressed by a piston, the
amount of work needed to reach a prescribed amount of compression depends on the type of
piston used (e.g.\ how hard or soft its surface is), and the pressure can be macroscopically
anisotropic and heterogeneous in absence of macroscopic stationary fluxes. 
%
We want to elucidate this within a many-body theory of ``dry active matter'' \cite{marchetti-etal:2013}, 
to which we successively add elements of ``wet reality'' \cite{zoettl-stark:2014}.

Similarly, the myriads of motile proteins and motors in a living cell cause deviations of the function
of its constituents with respect to conventional equilibrium predictions.  We call this
effect ``active crowding'', in analogy with the much studied ``molecular crowding''. 
%
Any biopolymer in a living cell actually represents a ``smart'' probe that scans its environment with its many 
internal degrees of freedom \cite{otto-etal:2013}.
%
We could recently demonstrate by theory and simulations that the effect of a quenched static crowding onto an embedded
polymer can effectively be captured by a parameter renormalization \cite{schoebl-etal:2014}.  
%
Here we ask: can something similar be achieved for active crowding?

The above fundamental issues severely limit a first-principles approach to living systems and
innovative technological applications involving active matter.
%
We address them as systematically as we can (though not comprehensively) with the above
objectives 1 \& 2 (theoretically) and 3 \& 4 (experimentally and by massively parallel
computer simulations).

\end{wpdescription}

\begin{tasklist}

\begin{task}[title=Non-Equilibrium Equations of State (NEOS),id=task1,PM=8,lead=Leipzig,
partners={Padova,USTUTT},wphases=0-48!0.3]
As a paradigm for a non-equilibrium medium we consider the (experimentally vast) class of
active-particle suspensions, in collaboration with the group in Padova providing expert knowledge on 
non-equilibrium response theory.
%
Our starting point is a many-body Langevin equation with a Markovian Gaussian noise
(describing the coupling to an equilibrium heat bath) and individual propulsion forces for
particles representing, on an all-atom-level, the "dry" active fluid far from equilibrium \cite{solon-etal:2015}. 
%
We let the fluid interact perturbatively with one large colloidal probe particle to
deduce its effective friction kernel, and with two or more probe particles in order to
detect their induced mutual interactions, non-gradient forces, and a possible breaking of
the action-reaction principle on the coarse-grained probe level.  
%
Non-equilibrium response theory lets us, in particular, include frenetic contributions in the response 
functions \cite{baiesi-wynants:2009}.
%
Of special interest to us is the mechanical pressure exerted by an active fluid on a probe, 
in particular if we relax the very special geometric symmetries tacitly imposed in previous approaches.

To add elements of reality, we then introduce more realistic particle-probe interactions, as appropriate for the hot 
Janus particles that we sudy in our experiments and simulations.
%
The results should help to clarify how active-particle suspensions are fundamentally different from externally driven systems, 
a question we want to elucidate in collaboration with Stuttgart, by searching   
the connection between our particle-based approach and their field theoretical approach.

\end{task}

\begin{task}[title=Active Crowding,id=task2,lead=Leipzig,wphases=0-48!0.2]
Pioneering attempts to estimate how the activity in a living cell affects the physical behavior
of its constituents predict that it will, via hydrodynamic interactions, enhance the diffusion of
passive globular probes or cell organelles \cite{mikhailov-kapral:2015}.
%
Yet, it remains open, how such partial results get modified if one considers, for example, that motors or
other sources of non-equilibrium do often act directly on cytoskeletal polymers and their
weak transient networks that may get disrupted.
%
In particular, we want to consider how three different types of active
crowding affect the fluctuation spectrum and dynamics of a semiflexible test polymer: a
laser-heated metal nanoparticle tracer attached to a micron-sized semiflexible polymer such
as an actin microfilament; power strokes exerted by attached molecular
motors and inducing backbone tension spikes along the polymer contour; the ``inelastic'' \cite{gralka-kroy:2015}
unbinding of weak bonds, which are the ubiquitous cement of
cytoskeletal structures, under static or dynamic loads.
%
This may have interesting conceptual consequences for mechanical homeostasis, i.e., how cells and tissues bear up against 
their own activity.   
%
On the technological side, it would be most useful to find conditions for a formal mapping to the situation in 
non-isothermal liquids, via frequency-dependent noise temperatures and effective friction coefficients.
%
For these tasks we can rely on our strong track record of first-principle analytical studies
of the non-linear non-equilibrium dynamics of biopolymers \cite{otto-etal:2013}, non-quasistatic bond kinetics
under strong external forcing \cite{bullerjahn-sturm-kroy:2014}, and hot Brownian motion \cite{rings-etal:2010}.

\end{task}


\begin{task}[title=Noise Temperature,id=task3,lead=Leipzig,wphases=0-24!0.5]
We have recently established a tractable ``drosophila'' for the central theme of the
proposal, i.e.\ probing macroscopic non-equilibria, in terms so-called non-isothermal (or
``hot'') Brownian motion \cite{rings-etal:2010,chakraborty-etal:2011}.
%
Its key quantity is a frequency-dependent
temperature for the Brownian noise strength (``noise temperature'') \cite{falasco-etal:2014}.
%
Corner stones of statistical mechanics, such as the fluctuation-dissipation theorem, can
thereby be generalized to conditions far from equilibrium.  

To test these predictions, which are relevant to metal-nanoparticle tracking and trapping technologies and emerging
nanoscopic Brownian heat engines, beyond the easily accessible Markov limit, requires to follow the fluctuations of 
a heated Brownian particle on a nanosecond timescale with a positional accuracy of picometers and below, a task 
that has, even under conventional \emph{equilibrium} conditions, only very recently been achieved \cite{kheifets-etal:2014}.
%
A dedicated optical trap with dual beam heating and a detection bandwidth of about 50 MHz
will therefore be newly constructed.

To test the theory also by massively parallel numerical simulations \cite{chakraborty-etal:2011}, 
we will extend the theory for hot nanoparticles in incompressible fluids 
to Lennard--Jones solvents, which exhibit a substantial compressibility.
\end{task}

\begin{task}[title=Active-Particle Suspensions,id=task4,lead=Leipzig,wphases=24-48!0.5]
Technological applications of active-particle suspensions and systematic tests of
uncontrolled approximations that are currently unavoidable in all theoretical studies are
currently severely hampered by a lack of control over these systems.
%
Therefore, dedicated experimental investigations are still very rare and some of their
results might be specific to the experimental realization of the self-propulsion mechanism.
%
We have recently established innovative force-free control systems, so-called photon nudging
and thermophoretic trapping technologies \cite{Qian2013,Braun:NanoLetters:2015}, which deliberately exploit 
the non-equilibrium transport caused by non-isothermal conditions.
%
Thereby we can e.g.\ construct soft walls for passive or active particles that act as
semi-permeable membranes.
%
Our experiments and accompanying "all-atom" Lennard--Jones molecular-dynamics simulations \cite{chakraborty-etal:2011} 
will employ hot Janus particles, which allow for a continuous tuning of the strength of 
the activity in space and time, to target 
possible failures of the description by a thermodynamic bulk-pressure equation of
state \cite{ginot-etal:2015} due to wall interactions and ensuing "ratcheting" effects.
\end{task}


\end{tasklist}

\printbibliography[heading=proposal-bib,env=proposal-env]

\begin{wpdelivs}
\begin{wpdeliv}[due=12,id=D2.3,dissem=PU,nature=DEM,lead=Leipzig]
      {Experimental set-up for noise temperature measurement}
  \end{wpdeliv}
  \begin{wpdeliv}[due=24,id=D2.1,dissem=PU,nature=DEM,lead=Leipzig]
      {Theoretical set-up for dynamics in active crowding}
  \end{wpdeliv}
  \begin{wpdeliv}[due=24,id=D2.2,dissem=PU,nature=DEM,lead=Leipzig]
      {Theoretical and numerical models for NEOS of active matter}
\end{wpdeliv}
  \begin{wpdeliv}[due=36,id=D2.4,dissem=PU,nature=DEM,lead=Leipzig]
      {Experimental setup for NEOS-test with hot Janus particles}
\end{wpdeliv}
\end{wpdelivs}

\end{workpackage}

\begin{workpackage}[id=WPcompress,wphases=0-48,
  short=Nonequilibrium compressibility, %XXX act. WP,% for Figure 5.
  title=Nonequilibrium compressibility, % XXX actual Work Package,
  lead=UNIPD,
  UNIPDRM=72,
  KULRM=12]

\newrefsection

\begin{wpobjectives}
  \begin{compactitem}
  \item Exploit the negative response to compression of systems holding a heat flow for the design of meta-materials.
  \item Apply effective fluctuation-response relations for the upscaling of industrial processes.
  \item Exhibit novel architectures of polymers and membranes under flowing conditions.
  \end{compactitem}
\end{wpobjectives}

\begin{wpdescription}

An important class of systems for which we need to understand the nonequilibrium response is that of solids 
in a temperature gradient. Examples include strongly
heated micro-cantilevers \cite{AGBB15} and high-quality table-top oscillators of gravitational wave detectors \cite{Cet13}.
In fact, compressibility of solids in a temperature gradient is a quite unexplored subject.
This WP plans to fill this gap, by understanding how the response to compression takes place and how it
may be predicted by just knowing suitable (unperturbed) steady-state correlations.
That knowledge is important for a second, main stage of this WP, as explained next.


The possibility to extract energy from an installed heat channel is the mechanism that we want to
analyze, to see if it can lead to the behaviour of negative differential compressibility,
i.e.~the system expands when compressed. 
Nonequilibrium response may be written as an entropic term minus a frenetic one \cite{BMW09}.
In our case the former is the correlation between the size of the system and the entropy produced by the compression.
The frenetic term is another correlation between size and dynamical aspects that can be monitored numerically.
This scheme thus furnishes us with a guideline to assess whether simulations of a specific material are yielding data 
going in the direction of negative compressibility. In particular, since typically the entropic term is positive, we
are encouraged to privilege the development of models that show a large frenetic term, so that its subtraction from
the entropic one gives a total negative response.
We have in this way delineated a guided trial-and-error strategy, in which the notions of 
nonequilibrium statistical mechanics help us in faster developments of numerical studies. This is expected to
speed up the research if compared to a blind trial-and-error strategy.


We can consider this WP as a high reward research plan. Indeed, relevant technological applications
may be boosted by a successful achievement in finding models of materials that have negative compressibility.
Metamaterials are nowadays much studied \cite{NM12,CG15} because, for example, artificial muscles,
next-generation pressure sensors, and micro-actuators would be
more easily realized by using materials with a negative compressibility. We will investigate
if steady nonequilibrium conditions could produce this effect, as opposed to the recently considered 
mechanism of rearrangements from metastable states \cite{NM12} or of enhanced expansions in orthogonal directions
due to a wine-rack structure of the materials \cite{CG15}. Thus, the anomalous response we seek is in the direction
of the compression and it is due to compression-enhanced extraction of energy from a heat channel,
which is prompt, excites the system expanding it, 
and does not require going through an hysteresis of the material. 

\printbibliography[heading=proposal-bib,env=proposal-env]

\end{wpdescription}

\begin{tasklist}

  \begin{task}[title=TASK1,id=task1,PM=3,lead=UNIPD,wphases={0-12!1,12-24!0.5}]
 
    Characterizing the response to compression
    for models of solids experiencing a heat flow due to different boundary temperatures.
    By applying available theoretical formulations to the study of numerical simulations,
    the main aim is to make a setup for the other tasks.
    
  \end{task}

  \begin{task}[title=TASK2,id=task2,PM=3,lead=UNIPD,partners=KUL,wphases={12-24!0.5,24-36!1}]

    From simulations and {\it ab initio} numerical integration, finding what microscopic details
    are needed  for
    predicting the response of the system's length to an increased compression.
    The main task of this WP is to
        find peculiar inter-particle potentials that may lead to negative 
        compressibility in solids experiencing heat flows.     
  \end{task}

  \begin{task}[title=TASK3,id=task3,PM=6,lead=UNIPD,wphases={18-24!0.5,24-48!1},partners={KUL,ULEI}]
Important probes for many industrial and medical applications are polymers, often found in nonequilibrium and/or nonlinear environments.
We set as task to give the typical shape and the architecture of these polymers, as chemical reactivity can much depend on that.
That will lead to improved control of these features via nonequilibrium monitoring.
It is important to reach guidelines for enhanced or selective reactivity, depending on the case, but for sure the topics of the workpackage and of the project in general are extremely important for understanding them.
There will collaborations here with the groups in Leuven (also with Enrico Carlon) and with the Leipzig node.  Regular visits and exchange lectures between Leuven and Padova are already in place.
  \end{task}

\end{tasklist}

\begin{wpdelivs}
  \begin{wpdeliv}[due=24,id=mydeliv1,dissem=PU,nature=DEM,lead=UNIPD]
      {First deliverable, after 2 years: characterization of compressibility for solids between two temperatures.}
  \end{wpdeliv}
  \begin{wpdeliv}[due=36,id=mydeliv2,dissem=PU,nature=DEM,lead=UNIPD]
      {Second deliverable, after 3 years: clear picture on how microscopic details are needed to predict compressibility from experiments}
  \end{wpdeliv}
  \begin{wpdeliv}[due=48,id=mydeliv3,dissem=PU,nature=DEM,lead=UNIPD]
      {Third deliverable, after 4 years: characterization of the compressibility for a wide class of heat-conduction models, understanding of whether negative compressibility is achievable in real systems.}
\end{wpdeliv}
\end{wpdelivs}




\end{workpackage}

\begin{workpackage}[id=WPbrown,wphases=0-48,
  short=Brown. particles, %XXX act. WP,% for Figure 5.
  title=Brownian particles in nonequilibrium baths, % XXX actual Work Package,
  lead=USTUTT,
  USTUTTRM=96,KULRM=6,ULEIRM=6,UNIPDRM=6]

\newrefsection

\begin{wpobjectives}
\begin{compactitem}
\item Characterization of nonequilibrium baths by driving probe particles in viscoelastic fluids
\item Experimental realization of self propulsion in natural (i.e. viscoelastic) media
  \end{compactitem}
\end{wpobjectives}

\begin{wpdescription}


Despite the success of stochastic control via Langevin dynamics \cite{blickle2006, blickle2007, blickle2012}, the assumption of a thermally equilibrated bath is not fulfilled in most industrially (oil, polymer melts, dense colloidal or micellar suspensions) or biologically relevant (blood, mucus, DNA-solutions) liquids which display visco-elastic properties, {\it i.e.}, elastic behavior like a solid and viscous
behavior like a fluid. As a consequence, such
fluids exhibit large relaxation times (up to seconds), and can thus easily be driven
out of thermal equilibrium by a forced colloidal probe. Within this proposal, we want to study how the local perturbation of visco-elastic systems affects the dynamics and effective interactions of colloidal particles in such media with particular view on e.g. memory effects, many body interactions and local non-linear rheological properties. Such studies will in general reveal the
properties of nonequilibrium baths which are of manifold importance. More explicitly, this approach will provide a rigorous test of the concepts of nonequilibrium thermodynamics (as outlined in the main
description).  It will be a crucial ingredient for the control and exploitation of processes where such nonlinear fluids are used. 


{\bf Specifically, in experiments} will study the dynamical behavior of {\bf externally driven} colloidal particles in different types of visco-elastic
fluids, e.g. micellar or polymer solutions. External forces will be created by optical, magnetic and gravitational forces where considerable expertise is present in the Stuttgart group. In addition to single driven particles, also complex, dynamical forces can be created by acousto-optical deflectors where controlled driving forces can be applied up to several hundred particles. In addition, the use of microfluidic devices, will allow us, to generate well-defined global flow fields \cite{scholz2012}, which will affect the visco-elastic properties of the liquid.

Furthermore, we will also study the motion of {bf self-propelled particles} in visco-elastic media. This will be achieved by a light-induced local demixing process which has been recently demonstrated by the Stuttgart group \cite{kuemmel2013, buttinoni2013, tenHagen2014} and which can be extended to visco-elastic systems. In contrast to externally driven particles, the force-field of self-propelled particles are fundamentally different. Accordingly, qualitative changes in the response of the visco-elastic liquids to self-propelled particles are expected. The comparision of the liquid's response to externally and self-driven particles will provide a more complete characterization of the
nonequilibrium fluctuations of a visco-elastic bath. One important aspect of this system is that the
probe, being on the micron scale (hence sensitive to fluctuations), is still large compared
to the constituents of the bath (sizes below $\sim 100$~nm), so that the desired {\it
  continuum description} of the bath can be expected to prove useful.

{\bf On the theoretical side}, our main goal will be deriving effective (continuum)
descriptions, ready for use for the modeling of dynamics of probes in visco-elastic fluids far from equilibrium. The effective description for the probe particle is
obtained via integration of bath degrees of freedom, for which several routes will be used,
including the Zwanzig-Mori projection formalism, as well as density functional
theory. The theoretical group in Stuttgart is very experienced in
such coarse graining procedures \cite{Aerov14} including  far-from-equilibrium physics \cite{Kruger11,Kruger09}.


{\bf In conclusion} we seek to provide practical guidelines and theoretical studies for controlling probes in nonlinear fluids and to characterize the latter via such probes.



\printbibliography[heading=proposal-bib,env=proposal-env]

\end{wpdescription}

\begin{tasklist}

\begin{task}[title=Experimental setup,id=brown-t1,PM=24,lead=USTUTT,wphases=0-24!0.5]
Implementation of an optical tweezers (including. acousto-optical deflectors), magnetic fields and a temperature- and flow-controlled sample cell in an optical microscope setup. Characterization of the visco-elastic properties of different systems (micellar, DNA solutions, polymers) with conventional rheometry and colloidal-probe rheology. 
\end{task}


\begin{task}[title=Externally driven particles in visco-elastic baths,id=brown-t2,PM=24,lead=USTUTT,wphases=0-24!0.5]
Study the response of externally driven colloids in visco-elastic liquids as a function of the driving amplitude, particle density, the presence of additional flows. As a function of driving, we will
measure mean (frictional) forces, translational fluctuations (diffusivities), rotational
fluctuations and nonequilibrium relaxation times. Application of more complex perturbation fields to the visco-elastic liquid by using magnetic, gravitational and scanned light fields (the latter achieved by acousto-optical modulators).
\end{task}

\begin{task}[title=Theoretical identification of nonequilibrium signatures of the bath,id=brown-t3,PM=24,lead=USTUTT,wphases=0-24!1.0,partners={KUL,UNIPD,ULEI}]
Direct comparison of nonequilibrium coarse graining procedures to experimental results will
determine the properties of theoretically emerging terms. Support via numerical simulations
(Padova) and effective medium theory (Leuven) will be crucial, and comparison to particle
based approaches (Leipzig) will be beneficial.
\end{task}

\begin{task}[title=Self-propelled particles in visco-elastic baths,id=brown-t4,PM=24,lead=USTUTT,wphases=24-48!0.5]
Experimental realization of critical mixtures with visco-elastic properties to achieve light-controlled active Brownian motion in such media. Measurement of single swimmer's trajectories in visco-elastic media and comparison with that of Newtonian systems. Study of gravitactic motion of spherical and non-spherical active particles in visco-elastic systems. 
\end{task}
\begin{task}[title=Nonequilbrium thermodynamics,id=brown-t5,PM=24,lead=USTUTT,wphases=24-48!1.0,partners={KUL,UNIPD,ULEI}]
Theoretical investigation of nonequilibrium thermodynamics for nonequilibrium baths, by
extracting dynamical activities and nonlinear response functions. Comparison of conclusions
to those of other work packages.
\end{task}

\begin{task}[title=Collective behavior of self-propelled particles in visco-elastic baths,id=brown-t6,PM=24,lead=USTUTT,wphases=24-48!0.5]
Investigation of particle interactions mediated by transient flow and strain fields, created by active particles. Consequences of such interactions for collective phenomena, like the formation of clusters. Structure formation in mixtures of active and passive particles in visco-elastic media.  
\end{task}


\end{tasklist}

\begin{wpdelivs}
  \begin{wpdeliv}[due=24,id=brown-d1,dissem=PU,nature=DEM,lead=USTUTT,miles=data1]
      {Experimental set-up for driven systems in visco-elastic media, characterization of (universal) properties of viscoelastic media}
  \end{wpdeliv}
  \begin{wpdeliv}[due=24,id=brown-d2,dissem=PU,nature=DEM,lead=USTUTT,miles=data1]
      {Theoretical description of nonequilibrium, visco-elastic baths based on microscopic starting points}
\end{wpdeliv}
 
  \begin{wpdeliv}[due=48,id=brown-d3,dissem=PU,nature=DEM,lead=USTUTT,miles=final]
       {Measurements of the response of self- driven particles in visco-elastic media}
 \end{wpdeliv}
\begin{wpdeliv}[due=48,id=brown-d4,dissem=PU,nature=DEM,lead=USTUTT,miles=final]
      {Theoretical analysis of nonequilibrium thermodynamics with regards to nonequilibrium baths}
\end{wpdeliv}
 \begin{wpdeliv}[due=48,id=brown-d5,dissem=PU,nature=DEM,lead=USTUTT,miles=final]
      {Measurements of the collective phenomena of driven particles and mixtures with passive particles in visco-elastic media}
\end{wpdeliv}
\end{wpdelivs}

\end{workpackage}

\begin{workpackage}[id=WPdissipation,wphases=0-48,
short=Dissipation,
title=Harnessing dissipation,
lead=TUE,
TUERM=42,
KULRM=6]

\newrefsection

\begin{wpobjectives}
\begin{compactitem}
\item Mathematical design of information-processing properties of nonequilibrium probes.
\item To identify the constructive and controllable elements in the time-symmetric part of transition rates that are influenced under nonequilibrium conditions.
\item Provide the mathematical basis for a prototype frenometer which would yield operational and experimentally accessible aspects of dynamical activity in the phenomenology of nonequilibrium thermodynamics.
\end{compactitem}
\end{wpobjectives}

\begin{wpdescription}
  % Overall description; typically 10 lines to half a page
  % as appropriate, depending on the variety and number of tasks
Recently it has become recognized that living cells make use of energy dissipation to reduce noise, increase sensitivity, and generally improve their information-processing capabilities. This is a non-trivial feat, since dissipation typically destroys correlations and eliminates information. Man-made microprobes and bioprobes could greatly benefit from this same possibility, provided we can understand and tune it appropriately. In addition, dissipation is energetically wasteful, which constitutes a problem for autonomous sensors. Careful control is necessary. 

In this project we therefore investigate the mechanisms and principles that underlie the beneficial sides of the usually harmful dissipation. The focus will be on extracting from examples the general guiding principles, allowing us to leverage this understanding for new applications.  In particular we will connect with the work of the Leuven-Padova-Prague nodes to identify the role of dynamical activity in response and fluctuation behavior.
Their constructions of frenometers (new word for the operational control and measurement of time-symmetric activity in nonequilibria) will be tested and confronted with specific situations of nonequilibrium thermodynamics.
Exchanges with Leuven will happen at a rate of visits of one day per month.

The work in this package will be possible by a combination of ingredients. We will leverage the link we recently discovered~\cite{AdamsDirrPeletierZimmer13} between large deviations and gradient flows to connect the macroscopic dissipation with microscopic features of the underlying dynamics. In addition, recent developments in chemical network theory give a tighter connection between stochastic dynamics and dissipation (e.g.~\cite{PolettiniWachtelEsposito15}). The work will also greatly benefit from the expertise of Leuven on nonequilibrium statistical physics and of Stuttgart on coarse-graining. 
Finally, the group in Eindhoven has an ongoing collaboration with an experimental group at the Institute for Complex Molecular Systems in Eindhoven, headed by Tom de Greef (see e.g.~\cite{RoekelMeijerMasroorGarzaEstevez-TorresRondelezZagarisPeletierHilbersGreef14}). This group focuses on bottom-up synthetic biology, and can provide essential chemical and biological insight. 

%
% test comments
%

% Sykes, A Guidebook to Mechanism in Organic Chemistry
\end{wpdescription}

% Please see UserInterfaces.tex for now as an example

\begin{tasklist}
  % 3-5 tasks

  % The description of each task can be 5 to 15 lines depending on the
  % complexity and amount of details deemed necessary, and involve and
  % refer to 1-3 deliverables.

  \begin{task}[title= Specification of dissipation in bistable switches,id=diss-t1,lead=TUE,wphases=0-12!1.0]
  We first focus on a single information-processing step, the bistable switch (a flip-flop), which is a core element of silicon-based computing, and which also exists in chemical form. For single examples the relation between dissipation and performance has been studied (e.g.~\cite{LanSartoriNeumannSourjikTu12,LanTu13,MehtaSchwab12,CaoWangOuyangTu15}). In this task we generalize from examples to a general principle that explains the underlying reasons.
  \end{task}

  \begin{task}[title=Operational control of constructive role of noise,id=diss-t2,lead=TUE,wphases=12-30!1.0,partners={KUL}]
  We next include time. An auto-nulling amplifier is a simple input-output information-processing unit, that temporarily amplifies differences in the input, and resets itself after some time. In earlier work (publication in preparation) we showed that such an object can be chemically constructed completely without dissipation, but only in a zero-noise context. We now include noise, and study how dissipation can be used to reduce the noise and improve the gain.  \end{task}
  
  \begin{task}[title=Solving toy-models for harnessing dissipation in operational units,id=diss-t3,lead=TUE,wphases=30-48!1.0]
A chemical oscillator is an intrinsically nonequilibrium object, that processes information by reacting in period and phase to external input. Dissipation is essential for the oscillator itself, and in this task we first study the role and limits provided by dissipation on the oscillator behaviour. We then continue with the effect of external inputs, and the role of dissipation in controlling this effect.    

\end{task}
  

\end{tasklist}

\printbibliography[heading=proposal-bib,env=proposal-env]

\eucommentary{ Deliverable numbers in order of delivery
  dates. Please use the numbering convention ``WP number''.``number of
  deliverable within that WP''.  For example, deliverable 4.2 would
  be the second deliverable from work package 4.
%
  Type:
  Use one of the following codes:
  R: Document, report (excluding the periodic and final reports)
  DEM: Demonstrator, pilot, prototype, plan designs
  DEC: Websites, patents filing, press \& media actions, videos, etc.
  OTHER: Software, technical diagram, etc.
  Dissemination level:
  Use one of the following codes:
  PU = Public, fully open, e.g. web
  CO = Confidential, restricted under conditions set out in Model Grant Agreement
  CI = Classified, information as referred to in Commission Decision 2001/844/EC.
  Delivery date
  Measured in months from the project start date (month 1)
}
\begin{wpdelivs}
  \begin{wpdeliv}[due=24,id=wp-diss-1,dissem=PU,nature=R,lead=TUE]
    {Mathematical road-map and characterization for the role of dissipation in the bistable switch.}
  \end{wpdeliv}
  \begin{wpdeliv}[due=36,id=wp-diss-2,dissem=PU,nature=DEM,lead=TUE]
    {DEM: Design of noise-control  in an auto-nulling amplifier.}
  \end{wpdeliv}
  \begin{wpdeliv}[due=48,id=wp-diss-3,dissem=PU,nature=R,lead=TUE]
    {Complete solution of the role of dissipation in nonequilibrium oscillator models.}
  \end{wpdeliv}
\end{wpdelivs}
\end{workpackage}

%%% Local Variables:
%%% mode: latex
%%% TeX-master: "../proposal"
%%% End:

\begin{workpackage}[id=WPcore,wphases=0-48,
  short=Gen. Theory, %XXX act. WP,% for Figure 5.
  title=General Theory, % XXX actual Work Package,
  lead=KUL,
  KULRM=36]

\begin{wpobjectives}

\end{wpobjectives}

\begin{wpdescription}

The Leuven node will have the role of coordinating and managing the network. That requires
not only some administrative tasks, and the (main-)organization of conferences, schooling or
workshops, but also the stimulation and inspiration of the work of the various other
nodes. The work in the Leuven and Prague nodes can then be called the most theoretical, as
major emphasis there will be on providing frameworks and conceptual schemes for
understanding and applying.

The core business of the Leuven and Prague nodes defining also this 6th workpackage in the
proposed network will be to provide a complement and extension of the irreversible
thermodynamics that has been developed in the previous century starting from Onsager's work
in the 1930's to the formulation of linear response theory 1960-70.  Major references from
around that time include the books by Mazur and de Groot (1964) and by Kubo (19??).
%
That theory put on a firm basis the nonequilibrium theory close-to-equilibrium, meaning in
the so called linear regime where driving forces are effectively small or where the initial
conditions are of local equilibrium type. Much technology has been based on the physics of
that irreversible thermodynamics, either for transport equations and the understanding of
linear response coefficients, or for a range of thermo-electric and thermo-magnetic
phenomena that have wide applications in industry.
%
We envision a radical step forward from this 50 year old basis. Time has come to incorporate
new ideas that have emerged over the last 20 years in the construction of a nonequilibrium
statistical mechanics, also away from equilibrium. In very general terms these developments
concern a fluctuation and response theory of systems beyond the linear regime around
reversibility. It reveals itself in constructive and predictive methods for higher order
response coefficients and for stabilization and possible control of effective dynamics in
nonequilibrium environments.
%
Modern control technologies, biomedical and food processing techniques and new directions in
materials research will undoubtly need to cross that boundary in the future. Workpackage 6
will provide the main resources on mathematical and theoretical levels to make free way for
the implementation of these techniques.

\end{wpdescription}

\begin{tasklist}

\begin{task}[title=Theory of statistical forces outside equilibrium,id=core-t1,PM=12,lead=KUL,wphases=0-24!0.5]
When probes get in contact with nonequilibrium media, their effective dynamics is governed
by forces that in general are no longer gradient (derivable from a potential), nor additive,
nor satisfying the action-reaction principle. Far from these being major setbacks, we
believe in turning these properties into new dynamical behavior showing unexpected phases of
matter.
\end{task}

\begin{task}[title=Stability and control theory,id=core-t2,PM=12,lead=KUL,wphases=12-36!0.5]
The theory of dynamical systems is playing a great role in robotics and cybernetics, and is
used in many applications with feedback mechanisms. We put this theory on a higher extended
level, where the control also includes nonequilibrium reservoirs that can steer effective
interactions of subsystems.
%
It allows further and different stabilization mechanisms which are needed when dealing with
nonequilibrium or strongly transient media.
\end{task}

\end{tasklist}

\begin{wpdelivs}
  \begin{wpdeliv}[due=24,id=core-d1,dissem=PU,nature=DEM,lead=KUL]
      {First deliverable, after 2 years}
  \end{wpdeliv}
  \begin{wpdeliv}[due=24,id=core-d2,dissem=PU,nature=DEM,lead=KUL]
      {Second deliverable, after 2 years}
\end{wpdeliv}
  \begin{wpdeliv}[due=48,id=core-d3,dissem=PU,nature=DEM,lead=KUL]
      {Third deliverable, after 4 years}
\end{wpdeliv}
\end{wpdelivs}

\end{workpackage}

\end{workplan}

%%% Local Variables:
%%% mode: latex
%%% TeX-master: "../proposal"
%%% End:


\subsection{Management Structure and Procedures}
\label{sect:mgt}




%%% Local Variables:
%%% mode: latex
%%% TeX-master: "proposal"
%%% End:


\draftpage
\subsection{Consortium as a Whole}

\eucommentary{\begin{compactitem}
\item
Describe the consortium. How will it match the project's objectives?
How do the members complement one another (and cover the value chain,
where appropriate)? In what way does each of them contribute to the
project? How will they be able to work effectively together?
\item
If applicable, describe the industrial/commercial involvement in the
project to ensure exploitation of the results and explain why this is
consistent with and will help to achieve the specific measures which
are proposed for exploitation of the results of the project (see section 2.3).
\item
Other countries: If one or more of the participants requesting EU funding
is based in a country that is not automatically eligible for such funding
(entities from Member States of the EU, from Associated Countries and
from one of the countries in the exhaustive list included in General
Annex A of the work programme are automatically eligible for EU funding),
 explain why the participation of the entity in question is essential to carrying out the project
\end{compactitem}
}

The consortium consists in a diversity of researchers, from the Czech republic, Germany, the
Netherlands, Italy and Belgium, all much engaged in international collaborations and
organizations.
%
The ages range between 55 and 35 years for the principal investigators with a large backbone
of researchers and infrastructure, allowing easy intake of students and postdocs. An equal
opportunity policy is maintained, with special emphasis on including excellent women
researchers in the team. For example, recent collaborations of the coordinator included also
supervision of and joint work with Soghra Safaverdi (woman from Teheran), with Simi Thomas
(woman student from India) and with Urna Basu (woman postdoc from India).  Members of the
consortium know each other from sharing the same objectives and from complimentary
expertise.
%
Applied mathematics, mathematical physics, theoretical physics, soft condensed matter and
liquid matter labs join here in a global effort around nonequilibrium physics, with the
special aim of developing knowledge and tools for control and manipulation of nonequilibrium
systems. Some members already work together, and have established research connections,
e.g. Leuven-Prague, Stuttgart-Leipzig and Leuven-Padova.  Other collaborations are more
recent, such as Leuven-Stuttgart and Padova-Leipzig.  Still other visits and discussions
started since about one year, Leuven-Eindhoven en Leuven-Leipzig.
%
There will continue to be many exchanges of students and postdocs and mutual visits to bring
the results forward.  In particular, important exchanges between the theoretical institutes
and experimental labs are foreseen.  Many of these members have contacts with research
institutes that allow easy access to industrial and commercial involvement.  For example,
the coordinator has collaborations and supervises students at imec (Leuven), one of the main
players in technological innovation in Europe. From November 2015 we will enter in direct
contact with the managers at imec for discussing new avenues in quantum technology, with the
damping of decoherence through contacts with nonequilibria as one of the new (and not te be
disclosed) possibilities for a major breakthrough.

%%% Local Variables:
%%% mode: latex
%%% TeX-master: "proposal"
%%% End:


\draftpage

\subsection{Resources to be Committed}
\eucommentary{Please provide the following:
\begin{compactitem}
\item
a table showing number of person/months required (table 3.4a)
\item
a table showing 'other direct costs' (table 3.4b) for participants where
those costs exceed 15\% of the personnel costs (according to the budget
table in section 3 of the administrative proposal forms)
\end{compactitem}}

\eucommentary{Please indicate the number of person/months over the whole
duration of the planned work, for each work package, for each participant.
Identify the work-package leader for each WP by showing the relevant
person-month figure in bold.}

The personnel resources are summarized in table~\ref{fig:staffeffort}.
%
For \site{KUL}, \site{TUE} and \site{UNIPD} the effort is theoretical while for \site{ULEI}
and \site{USTUTT} there is a strong experimental effort.

\wpfig[label=fig:staffeffort,caption=Summary of Staff Efforts.]

The ``other direct costs'' in excess of 15\% of the ``personnel direct cost'', displayed in
table~\ref{fig:otherdirect}, are related to experimental material for \site{ULEI} and
\site{USTUTT} and to the comparatively reasonable cost of work in Italy for \site{UNIPD},
where the actual amount is similar to that of the other partners.

\begin{table}[h]
\centering
\begin{tabular}{|l||l|l|}\hline%
Node & Amount & Nature\\\hline\hline%
\site{UNIPD} & 29 k~EUR & Travel\\
\hline
 & 15 k~EUR & Computers\\
\hline
 & 8 k~EUR & Project meeting\\
\hline
\site{USTUTT} & 18 k~EUR & Travel\\
\hline
& 34 k~EUR & Consumables\\
\hline
& 34.699 EUR & Experimental apparatus\\
\hline
& 8 k~EUR & Project meeting\\
\hline
& 5 k~EUR & Audit\\
\hline
\site{ULEI} & 18 k~EUR & Travel\\
\hline
& 25 k~EUR & Consumables\\
\hline
& 34 k~EUR & Experimental apparatus\\
\hline
& 8 k~EUR & Project meeting\\
\hline
& 5 k~EUR & Audit\\
\hline
\end{tabular}
\caption{Other direct costs, for partners exceeding 15\% of personnel cost. \label{fig:otherdirect}}
\end{table}

%%% Local Variables:
%%% mode: latex
%%% TeX-master: "proposal"
%%% End:


% ---------------------------------------------------------------------------
%  Section 4: Members of the Consortium
% ---------------------------------------------------------------------------

\newpage

\eucommentary{This section is not covered by the page limit.\\
The information provided here will be used to judge the operational capacity.}

\section{Members of the Consortium}

\subsection{Participants}

\eucommentary{Please provide, for each participant, the following (if available):\\
\begin{compactitem}
\item
a description of the legal entity and its main tasks,
with an explanation of how its profile matches the tasks in the proposal;
\item
a curriculum vitae or description of the profile of the persons,
including their gender, who will be primarily responsible for carrying
out the proposed research and/or innovation activities;
%
this includes a description of the profile of the to-be-recruited personnel
\item
a list of up to 5 relevant publications, and/or products, services
(including widely-used datasets or software), or other achievements
relevant to the call content;
\item
a list of up to 5 relevant previous projects or activities, connected
to the subject of this proposal;
\item
a description of any significant infrastructure and/or any major items
of technical equipment, relevant to the proposed work;
\item
any other supporting documents specified in the work programme for this call.
\end{compactitem}}

\begin{sitedescription}{KUL} \label{desc:KUL}

KU Leuven is currently by far the largest university in Belgium in terms of
research funding and expenditure (EUR 426.5 million in 2014), and is a charter
member of LERU. KU Leuven conducts fundamental and applied research in all
academic disciplines with a clear international orientation.  Leuven
participates in over 540 highly competitive European research projects (FP7,
2007-2013), ranking sixth in the league of HES institutions participating in
FP7. In Horizon 2020, KU Leuven currently has been approved 79 projects.

KU Leuven takes up the 9th place of European institutions hosting ERC grants (as
first legal signatories of the grant agreement). To date, the
\href{http://www.kuleuven.be/english/research/EU/p/erc}{78 ERC Grantees}
(including affiliates with VIB and IMEC) in our midst confirm that KU Leuven is
a breeding ground (51 Starting Grants) and attractive destination for the
world's best researchers. The success in the FP7 and Horizon 2020 Marie
Sklodowska Curie Actions is a manifestation of the three pillars of KU Leuven:
research, education and service to society. In our
\href{http://www.kuleuven.be/english/research/EU/p/horizon2020/es/msca}{170
Actions}, of which 76 Initial/European Training Networks, hundreds of young
researchers have been trained through research and have acquired the necessary
skills to transfer their knowledge into the world outside academia.
%
KU Leuven Research \& Development (LRD) is the technology
transfer office (TTO) of the KU Leuven. Since 1972 a multidisciplinary team of
experts guides researchers in their interaction with industry and society, and
the valorisation of their research results (101 spin offs, \dots).

Within the KU Leuven, the Institute for Theoretical Physics (ITF) has 8 permanent staff
members and about 15 PhD students and 10 postdocs in three different areas of Modern
Theoretical Physics: High-Energy Physics, Mathematical Physics and Statistical Physics.
%
The support staff of the ITF (one secretary and one IT support person) ensures excellent
working conditions and, in combination with the administration of the KU Leuven, provides an
ideal environment for developing ambitious research projects.
%
The Department of Physics is host to many excellent researchers and six of its professors
are currently running a ERC grant.

The ITF enjoys regular contacts with the nearby IMEC, a research institute dedicated to
nanoelectronics. Prof. Christian Maes supervises a student jointly with IMEC.

\subsubsection*{Curriculum vitae of the investigators}

\begin{participant}[type=PI,PM=12,gender=male,salary=5500]{Christian Maes}
\url{http://fys.kuleuven.be/itf/staff/christ}

Full professor at the KU Leuven and director of the Institute for Theoretical Physics.

Christian Maes is a leading scientist in the field of mathematical and nonequilibrium statistical mechanics, regularly
invited as a keynote to scientific events worldwide.
%
He has published 150 articles in peer-reviewed journals,
is currently an associate editor or member of the editorial board of 4 leading international journals,
has supervised 14 PhD theses (2 more ongoing) and 11 postdoctoral researchers,
is expert and reviewer for many scientific institutions and
is a member of various international evaluation commissions.

\end{participant}

%%% Local Variables:
%%% mode: latex
%%% TeX-master: "../proposal"
%%% End:

\begin{participant}[type=R,PM=12,gender=male,salary=5500]{Pierre de Buyl}

Postdoctoral researcher at the KU Leuven.
Pierre de Buyl holds a PhD in Physics (2010) and is specialized in theoretical and
computational studies in statistical physics. He has published 15 papers in international
peer-reviewed journals and also made conference contributions. He has worked at the
Université libre de Bruxelles, the University of Toronto and is now at the KUL.
%
His early work is about the consequences of long-ranged interactions on the dynamical
evolution of models relating to plasmas and gravitation.
%
His current focus is on so-called nanomotors, a class of devices that transforms a fuel into
motion for the motor (hence the name). The operation conditions are fluctuating and stronly
out of equilibrium.

In addition to his research skills, de Buyl also participates in the organization of
scientific events (web site and database for the {\em European Conference on Complex Systems
  2012}, organizer of the conferences {\em EuroSciPy 2012} and {\em EuroSciPy 2013},
webmaster for EuroSciPy since 2013 and proceedings editor in 2013 and 2014).

\end{participant}


\begin{participant}[type=res,PM=48,salary=5500]{NN}
A postdoc will be hired to work on the project. We aim to hire someone who has a strong
background in theoretical statistical mechanics and a diverse research experience
(applications, computations, etc).
%
This person will be in contact with several other nodes and should be willing to engage in a
large collaborative effort.
\end{participant}

\begin{participant}[type=res,PM=48,salary=3500]{NN}
A PhD student will be hired for the duration of the project that matches the duration of a
PhD thesis in Belgium. The ITF graduates students with an advanced knowledge of statistical
mechanics, ensuring that skilled candidates will exist for the position.
%
The opening will indeed be international and open to competent candidates from any
origin.
\end{participant}

\begin{participant}[type=res,PM=24,salary=3932]{NN}
We will hire an experienced part time project manager to help with the overall management
during the whole duration of \TheProject.
\end{participant}

\subsubsection*{Publications, achievements}

\begin{compactenum}
\item M. Baiesi, C. Maes and B. Wynants {\em Fluctuations and response of nonequilibrium
  states}, Physical Review Letters {\bf 103}, 010602 (2009). This article is already cited
137 times according to Google Scholar, an achievement for a theoretical work in statistical
physics. This article highlights the increasing interest in nonequilibrium physics.
\item {\tt vmf90} software for the numerical resolution of the Vlasov equation:
\url{https://github.com/pdebuyl/vmf90}. This demonstrates the applicant's ability to write
open-source software of scientific relevance.
\end{compactenum}


\subsubsection*{Previous projects or activities}

\begin{compactenum}
\item Organization of the international school ``Fundamental problems in Statistical
Physics''.
%
This school is organized every four years since the 1970s (since 2005 in Leuven) and brings
together the world's most influential experts on Statistical Physics.
\item Membership of expert committees for the Irisch Research Council, the ERC Starting
Grants in Mathematics, of the steering committee of the European Science Foundation
Programme Random Geometry of Large Interacting Systems and Statistical Physics (RGLIS),
among others.
\item Partner in national collaborative projects (Belgian federal government).
\end{compactenum}

\end{sitedescription}



\begin{draft}
\vspace{1cm}\TOWRITE{PAR1P1}{Complete check list below -- delete completed items if you wish}

\begin{verbatim}
- [ ] checked that sum of person months put into finance request is
  the same as sum of person months associated with the Work Packages
  (in proposal.tex, as defined as part of the \begin{workpackage}"
  command.
  
- [ ] completed site specific resource summary in resources.tex,
  including table of non-staff costs.

\end{verbatim}
\end{draft}

%%% Local Variables: 
%%% mode: latex
%%% TeX-master: "../proposal"
%%% End: 

\begin{sitedescription}{TUE}

TUE is an internationally leading research university specialised in engineering science and technology. TUE is a constant presence in the top of various rankings, all acknowledging in particular the research carried out in collaboration with the industry, and currently manages more than 150 EU-funded projects. TUE offers 50 different educational programs to about 9000 students of various levels, and employs more than 3000 academic and administrative personnel.

The TUE research team in this proposal is part of both the Department of Mathematics and Computer Science and the Institute for Complex Molecular Systems. 
The Department Mathematics and Computer Science (MCS) unites all activities on campus in the classical scientific disciplines of mathematics and computer science. Both research and education have recently been evaluated as excellent by international committees. 40\% of the employees at MCS are of non-Dutch origin, thus creating a truly multi-cultural environment. 

The Institute for Complex Molecular Systems (ICMS) is a hotbed for interdisciplinary interaction in research and education across the university. Its mission is to become the leading international multidisciplinary institute for research and education in the area of the engineering of complex molecular systems.

The joint affiliation with both the Department of Mathematics and Computer Sciences and the ICMS provides the TUE research team with both excellent mathematical infrastructure and broad experimental embedding.


\subsubsection*{Curriculum vitae}

% Curriculum of the personnel at this institution. This includes
% to-be-hired people for which there is a tentative candidate.

\begin{participant}[type=leadPI,PM=24,gender=male,salary=5500]{Mark Peletier}

Professor at the Technische Universiteit Eindhoven.

\end{participant}

%%% Local Variables:
%%% mode: latex
%%% TeX-master: "../proposal"
%%% End:


\begin{participant}[type=R,PM=12,gender=male,salary=5500]{Adrian Muntean}


Adrian Muntean is lecturer at the Center for Analysis, Scientific Computing, and Applications and the Institute for Complex Molecular Systems at TUE. His research interests lie in 
the theoretical understanding of basic interactions between large scale transport and local interactions, such as phase transitions, chemical reactions, physical/social collisions, through separated length scales ranging from discrete and stochastic systems to multiscale continua. 

With cum laude PhD student Joep Evers (2015) he has contributed to the mathematical  understanding of models for crowd evolution, including the essential interaction with hard and soft boundaries. Other objects of study are defects in non-homogeneous materials and ``swimming'' of active particles in compressible flows.

He has published more than 60 papers in peer-reviewed journals, edited two books and five special issues, supervised 10 master students and five PhD students, and organized more than 20 conferences and workshops. 

\end{participant}

%%% Local Variables:
%%% mode: latex
%%% TeX-master: "../proposal"
%%% End:

% For other to-be-hired person, please include here something like:
% \begin{participant}[type=res,PM=3,salary=5900]{NN}
%  <a _short_ description of the qualifications of whom you want to hire>
% \end{participant}

\begin{participant}[type=res,PM=48]
A PhD student will be hired for the duration of the project that matches the duration of a
PhD thesis in the Netherlands. 
%
The opening will be international and open to competent candidates from any
origin.
\end{participant}


\subsubsection*{Publications, products, achievements}

\begin{compactenum}
\item {In a series of papers, starting with \emph{Adams et al., Communications in Mathematical Physics, 307:791 (2011)}, Peletier and co-authors identified and explored the deep relations between gradient flows on one hand and large deviations of stochastic processes on the other. These relations create new understanding of the mathematical structures describing physical phenomena at the mesoscale, and point the way towards understanding the strongly nonequilbrium systems of this proposal.}
\item {M. A. Peletier, G. Savar\'e, and M. Veneroni, From diffusion to reaction via $\Gamma$-convergence, SIAM Journal on Mathematical Analysis, 42(4), pp. 1805--1825, 2010.
This paper has sparked renewed interest in the analysis community in the old problem of a stochastic particle escaping from a potential well, by introducing a new method to combine reactive and diffusive effects. It was selected as a SIGEST paper in SIAM Review 54(2) in 2012.}
\item {Peletier is a co-founder and board member of the ICMS, and both Peletier and Muntean are prominent members of this interdisciplinary institute.}
\end{compactenum}

\subsubsection*{Previous projects or activities}

\begin{compactenum}
\item {Peletier is and has been PI on many national and international projects, including an EU Seventh-framework ITN project `Fronts and Interfaces in Science and Technology'.}
\item {Peletier and Muntean have together and separately organized over 50 meetings, workshops, and conferences.}
\end{compactenum}

\subsubsection*{Significant infrastructure}

{Both CASA and the ICMS have significant in-house computing resources (several computing clusters) as well as access to national computing facilities at SARA (Amsterdam).}
\end{sitedescription}

%%% Local Variables:
%%% mode: latex
%%% TeX-master: "../proposal"
%%% End:

\begin{sitedescription}https://github.com/pdebuyl-lab/fet-open-promane-2015/edit/master/Participants/Leipzig.tex#{Leipzig} \label{desc:Leipzig}

 Universität Leipzig was founded in 1409 and is the second oldest university in Germany where teaching has continued without interruption. 
  Today it offers a wide spectrum of academic disciplines at 14 faculties with more than 150 institutes. 
  It is a member of the German U15, a strategic alliance of 15 major German research universities. 
  The node participants are located at the Institute for Experimental Physics I and at the Institute for Theoretical Physics, respectively, which are parts of the Faculty of Physics and Earth Sciences. 
  The faculty, which enroles more than 1200 students from more than 30 countries and offers an international study program for physics, also comprises institutes for meteorology, geology, geophysics and geography. 
  It counts among the leading faculties in terms of research output and external funding, within the university, and hosts several ERC grantees. 
  It is also the first faculty of the university that recently got its research activities evaluated by an external board of international experts. 
  The physics institutes make major contributions to several collaborative research centers (SFBs) funded by the German Science Foundation (DFG) and to the interdisciplinary Graduate School ``Leipzig School of Natural Sciences -- Building with Molecules and Nano-objects'' (www.buildmona.de), founded by a grant from the German Excellence Initiative, which has so far enrolled close to 200 PhD candidates. 
  They maintain collaborations and joint grants with a large number of independent international and local research institutes for fundamental and applied science and industrial partners. 

\subsubsection*{Curriculum vitae of the investigators}

\begin{participant}[type=R,PM=12,gender=male,salary=5500]{Klaus Kroy}

Professor at the Institute of Theoretical Physics, Universität Leipzig.

Klaus Kroy is a theoretical physicist and an expert in the field of
soft mesoscopics (nonequilibrium dynamics of colloids and polymers; active particles; cytoskeleton and tissue mechanics; single-molecule force spectroscopy; aeolian sand transport and structure formation)

He has published about 60 articles in peer-reviewed journals (also in
Nature Physics, Nature Communications, PNAS, PRL) 

He has in the past supervised 3 postdocs, 5 PhD students, and 19
master students 

He is a Member of the German Physical Society, of the International Max--Planck Research Group
Mathematics in the Sciences (Leipzig), and he recently received grants from
the German Excellence Initiative (Graduate School ``BuildMoNa''),
the DFG-Forschergruppe FOR 877, the German
priority programm SPP1726 (DFG), the German Israel
Foundation, the ESF, and a DFG-individual-grant.


\end{participant}

\begin{participant}[type=R,PM=12,gender=male,salary=5500]{Frank Cichos}

Professor at the Institute of Experimental Physics I, Universität Leipzig. \url{http://www.uni-leipzig.de/~mona}

Frank Cichos is an experimental physicist and an expert optical microscopy and optical single molecule detection (photothermal single molecule detection; active particles; single molecule trapping; single molecule dynamics in soft matter)

He has published about 76 articles in peer-reviewed journals (also in
Nano Letters, ACS nano and PRL) 

He has in the past supervised 2 postdocs, 15 PhD students, and about 20
master students.

He is a Member of the German Physical Society and the American Physical Society. He recently received grants from the German Excellence Initiative (Graduate School ``BuildMoNa''), the DFG-Forschergruppe FOR 877, the German priority program SPP1726 (DFG), the DFG Sonderforschungsbereich TRR102 and a joint  DFG-ANR-individual-grant. He is the co-speaker of the DFG Sonderforschungsbereich TRR102 and has been the speaker of the DFG-Forschergruppe FOR 877.

\end{participant}


\begin{participant}[type=res,PM=48,salary=5500]{NN}
\end{participant}
\begin{participant}[type=res,PM=36,salary=5500]{NN}

We need researchers. Two, for instance.

\end{participant}

\subsubsection*{Publications, achievements}

\begin{compactenum}
\item Leadership.
M. Gralka, K. Kroy, Inelastic mechanics: A unifying principle in biomechanics. 
Biochimica et Biophysica Acta (BBA)--Molecular Cell Research 2015

S. Schöbl, S. Sturm, W. Janke, K. Kroy, Persistence-Length Renormalization of Polymers in a Crowded Environment of Hard Disks.
Physical Review Letters 113 (2014) 238302.

J. T. Bullerjahn, S. Sturm, K. Kroy, Theory of rapid force spectroscopy. Nature Communications 5 (2014) 4463.

O. Otto, S. Sturm, N. Laohakunakorn, U. F. Keyser, K. Kroy, Rapid internal contraction boosts DNA friction.
Nature Communications 4 (2013) 1780.

D. Chakraborty, M. V. Gnann, D. Rings, J. Glaser, F. Otto, F. Cichos, K. Kroy, 
Generalised Einstein relation for hot Brownian motion. EPL (Europhysics Letters) 96 (2011) 60009.

\item Coauthoring.
\end{compactenum}

\subsubsection*{Previous projects or activities}

\begin{compactenum}
\item Organization.
\item Partner.
\end{compactenum}

\subsubsection*{Significant infrastructure}


\end{sitedescription}

\begin{draft}
\vspace{1cm}\TOWRITE{PAR1P1}{Complete check list below -- delete completed items if you wish}

\begin{verbatim}
- [ ] checked that sum of person months put into finance request is
  the same as sum of person months associated with the Work Packages
  (in proposal.tex, as defined as part of the \begin{workpackage}"
  command.
  
- [ ] completed site specific resource summary in resources.tex,
  including table of non-staff costs.

\end{verbatim}
\end{draft}

%%% Local Variables: 
%%% mode: latex
%%% TeX-master: "../proposal"
%%% End: 

\begin{sitedescription}{Padova} \label{desc:Padova}

The first ever report issued by the National Research Assessment Committee set the university of Padova as the highest ranking among leading Italian universities for the quality of its research results.
Twenty panels of evaluators, including national and international experts (25\% of whom come from foreign institutions) analysed the scientific articles, monographs, patents, works and publications presented by 77 universities, 12 public bodies and 13 private research institutes from all over Italy before awarding the University of Padova a vote of excellence in 12 out of the 20 disciplines and scientific areas covered.
In particular, the Department of Physics and Astronomy ``Galileo Galilei'' was ranked first among the physics departments of large dimension in Italy.

Based on the number of citations of articles and publications by its researchers (source: ISI), the University of Padova ranks among the top three Italian universities for total impact index, productivity index, and presence index.
Research at Padova is also attracting more and more public and private funding. Some 5.2\% of the resources that the Italian State allocates to scientific projects of national importance are awarded to Researchers at the University of Padova, while financing from the European Union for projects in various departments and research centres at the university has grown by 40\% over the last three years.
Further confirmation of the university’s ability to contribute to the cultural development and economic growth of the region comes from the number of ongoing contracts with public and private bodies for experimental activities. Commercial research services alone account for 32\% of the income of the university’s departments.

The University of Padova manages more than 200 European research projects. In 2013, 29 new projects were approved by the European Commission within the FP7. The total sum of EU contribution assigned to those projects amounts to more than 11 Million Euro.
The participation of the University of Padova in the Seventh Framework Programme has then increased: 194 is now the number of funded projects within the FP7 framework, out of which 41 managed as a coordinating body, for a total contribution of € 65 Million. Among them, 15 projects were funded by the European Research Council within the specific programme “Ideas”.
The successful participation of the University of Padova in this programme in 2012 and 2013 represents the best result among the Italian Universities. “Ideas” is the most eminent EU programme for research funding. It supports projects with a contribution that varies from 1.5 to 3.5 Million Euro. Furthermore, the University of Padova also performed in a distinguished way in other EU programmes in the last years for a total EU contribution of 3.5 Million Euro.


\subsubsection*{Curriculum vitae of the investigators}

\begin{participant}[type=PI,PM=12,gender=male,salary=2500]{Marco Baiesi}

  Marco Baiesi ....

\end{participant}

%%% Local Variables:
%%% mode: latex
%%% TeX-master: "../proposal"
%%% End:

\begin{participant}[type=PI,PM=8,gender=male,salary=2500]{? ??}

Attilio L. Stella ...

\end{participant}

%%% Local Variables:
%%% mode: latex
%%% TeX-master: "../proposal"
%%% End:


\begin{participant}[type=res,PM=48,salary=2500]{NN}
\end{participant}
\begin{participant}[type=res,PM=36,salary=2500]{NN}

We need researchers. Two, for instance.

\end{participant}

\begin{participant}[type=res,PM=24,salary=3932]{NN}
  We will hire an experienced part time project manager to help with
  the overall management during the whole duration of \TheProject.
\end{participant}

\subsubsection*{Publications, achievements}

\begin{compactenum}
\item Leadership.
\item Coauthoring.
\end{compactenum}


\subsubsection*{Previous projects or activities}

\begin{compactenum}
\item Hosting.
\item Co-organising.
\end{compactenum}

\subsubsection*{Significant infrastructure}

We have building, at PAR????.

\end{sitedescription}



\begin{draft}
\vspace{1cm}\TOWRITE{PAR1P1}{Complete check list below -- delete completed items if you wish}

\begin{verbatim}
- [ ] checked that sum of person months put into finance request is
  the same as sum of person months associated with the Work Packages
  (in proposal.tex, as defined as part of the \begin{workpackage}"
  command.
  
- [ ] completed site specific resource summary in resources.tex,
  including table of non-staff costs.

\end{verbatim}
\end{draft}

%%% Local Variables: 
%%% mode: latex
%%% TeX-master: "../proposal"
%%% End: 

\begin{sitedescription}{FZU}

Fyzikální Ústav AV ČR, v. v. i. (FZU; in English: Institute of Physics of the Czech Academy
of Sciences) is a public research institute, oriented on the fundamental and applied
research in physics.

\subsubsection*{Curriculum vitae}

\begin{participant}[type=leadPI,PM=24,gender=male,salary=5500]{Karel Netočný}

Scientist in the division of Condensed Matter Physics (2).

\end{participant}

%%% Local Variables:
%%% mode: latex
%%% TeX-master: "../proposal"
%%% End:


\subsubsection*{Publications, products, achievements}

\begin{compactenum}
\item \TOWRITE{XXX}{...}
\end{compactenum}

\subsubsection*{Previous projects or activities}

\begin{compactenum}
\item \TOWRITE{XXX}{...}
\end{compactenum}

\subsubsection*{Significant infrastructure}

\TOWRITE{XXX}{...}
\end{sitedescription}

%%% Local Variables:
%%% mode: latex
%%% TeX-master: "../proposal"
%%% End:

\begin{sitedescription}{USTUTT} \label{desc:USTUTT}

{\bf Universität Stuttgart -- A Research University of International Standing:}\\
The Universität Stuttgart lies right in the centre of the largest high-tech region of Europe. We are surrounded by a number of renowned research facilities and have such global players as Daimler or IBM as our neighbours. We were founded in 1829 and over the years this technical institution has developed to the research intensive university that it is today. Our main emphasis is on engineering and the natural sciences.
%
Indicators of our excellent status are the two projects that were successful in the recent {\it Excellence Initiative} sponsored by both the Federal and the State governments. One project is the Cluster of Excellence {\it Simulation Technology} and the other, the Graduate School {\it Advanced Manufacturing Engineering}.

{\bf Experience with EU research funding:}\\
The University of Stuttgart has extensive experience with the various funding programs of the European Commission and has been the {\it leading German university in FP6}, both in number of projects (184) and in terms of funding (54 Mio. \euro). In FP7, it was yet again {\it among the most successful German universities} with 246 projects funded and a total budget of 94 Mio. \euro. The University has consistently been involved in Marie-Curie-projects in previous framework programs. In FP7, it participated in 13 Marie Curie actions.

\subsubsection*{Curriculum vitae of the investigators}

\begin{participant}[type=R,PM=12,gender=male,salary=5500]{Matthias Krüger}

Research Group Leader at the MPI Intelligent Systems.

\end{participant}

\begin{participant}[type=PI,PM=12,gender=male,salary=5500]{Clemens Bechinger}

Full professor and head of the 2nd Experimental Institute at the University
of Stuttgart. Fellow of the Max Planck Institute of Intelligent Systems.

%

Clemens Bechinger is an expert in the field of experimental soft matter
systems and regularly invited as keynote and plenary speaker in
international scientific meetings.

%

He has published about 125 articles in peer-reviewed journals (also in
Nature, Science, PNAS, PRL) and is member of the liquid matter board of the
EPS and the Panel ``Statistical Physics, Soft Matter, Biophysics, Nonlinear
Dynamics'' of the German Research Society. Since his arrival in Stuttgart in
2003 he has supervised 10 Postdocs, 18 Phd Students and 22 Master Students.

\end{participant}

\begin{participant}[type=res,PM=48,salary=5500]{NN}
Postdoctoral Researcher.
\end{participant}
\begin{participant}[type=res,PM=36,salary=5500]{NN}
PhD Student.
\end{participant}

\subsubsection*{Publications, achievements}

\begin{compactenum}
\item Leadership.
\item Coauthoring.
\end{compactenum}

\subsubsection*{Previous projects or activities}

\begin{compactenum}
\item Organization.
\item Partner.
\end{compactenum}

\subsubsection*{Significant infrastructure}

\end{sitedescription}


\subsection{Third Parties Involved in the Project (including use of third party resources)}
\label{section:ThirdParties}

\paragraph{Third Party 1}\ 

\eucommentary{Please complete, for each participant, the table
(see page 27 of "VRETemplate.PDF"),
or simply state "No third parties involved", if applicable.}

Third Party 1 (hereafter TP1) will work on the project.

\paragraph{Other participants}\ 

For other participants, the only subcontracting costs will be for audit.

\bgroup
\def\arraystretch{1.5}  % 1 is the default
\noindent \begin{tabular}{|p{0.6\textwidth}|c|}
\hline
Does the participant plan to subcontract certain
tasks & Yes \\
\hline
\multicolumn{2}{|l|}{Audit} \\
\hline
Does the participant envisage that part of its work
is performed by linked third parties & No \\
\hline
\multicolumn{2}{|l|}{} \\
\hline
Does the participant envisage the use of
contributions in kind provided by
third parties & No \\
\hline
\multicolumn{2}{|l|}{} \\
\hline
\end{tabular}
\egroup

%No third parties involved.

% ---------------------------------------------------------------------------
%  Section 5: Ethics and Security
% ---------------------------------------------------------------------------

\newpage

\section{Ethics and Security}

\eucommentary{This section is not covered by the page limit.}

\subsection{Ethics}

\eucommentary{
If you have entered any ethics issues in the ethical issue table in the administrative proposal forms, you must:\\
$\bullet$ submit an ethics self-assessment, which: \\
-- describes how the proposal meets the national legal and ethical requirements of the
country or countries where the tasks raising ethical issues are to be carried out;\\
-- explains in detail how you intend to address the issues in the ethical issues table, in
particular as regards:
research objectives (e.g. study of vulnerable populations, dual use, etc.),
research methodology (e.g. clinical trials, involvement of children and related
consent procedures, protection of any data collected, etc.),
the potential impact of the research (e.g. dual use issues, environmental damage,
stigmatisation of particular social groups, political or financial retaliation,
benefit-sharing, malevolent use , etc.)\\
$\bullet$ provide the documents that you need under national law (if you already have them), e.g.:\\
-- an ethics committee opinion;\\
-- the document notifying activities raising ethical issues or authorizing such activities\\
If these documents are not in English, you must also submit an English summary of them
(containing, if available, the conclusions of the committee or authority concerned).\\
If you plan to request these documents specifically for the project
you are proposing, your request must contain an explicit reference to the project title}

\subsection{Security}

Please indicate if your proposal will involve:

\begin{compactitem}
\item activities or results raising security issues: NO
\item 'EU-classified information' as background or results: NO
\end{compactitem}
\end{proposal}
\TOWRITE{ALL}{Search through final.pdf ('make final') and look for questions marks ?? and XX and YY and XYZ as place holders where people intended to later add a link, or where a link is broken.}
\end{document}

%%% Local Variables:
%%% mode: latex
%%% TeX-master: t
%%% End:

