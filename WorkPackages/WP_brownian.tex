\begin{workpackage}[id=WPbrown,wphases=0-48,
  short=Brown. particles, %XXX act. WP,% for Figure 5.
  title=Brownian particles in nonequilibrium baths, % XXX actual Work Package,
  lead=USTUTT,
  USTUTTRM=72,KULRM=6,LeipzigRM=6,PadovaRM=6]

\begin{wpobjectives}
The objectives of this WP are:
\begin{compactitem}
\item Experimental characterization of non-equilibrium baths by driving probe particles in viscoelastic solvents
\item Experimental realization of self propulsion in natural (i.e. viscoelastic) media
\item Theoretical description of non-equilibrium baths from microscopic starting points
\item Theoretical and experimental analysis of non-equilibrium thermodynamics with regards to non-equilibrium baths
  \end{compactitem}
\end{wpobjectives}

\begin{wpdescription}

Concerning non-equilibrium statistical physics, in recent years a lot of experimental and
theoretical insight has been gained by studying colloidal (Brownian) particles or
suspensions. Such particles, being in the size range of microns, are in many aspects ideal
model systems; They are almost 'perfect' random walkers, hence described very accurately by
simple stochastic equations (so called Langevin equations). Most notably, such equations
safely assume that the solvent (i.e., the bath) is in {\it thermal equilibrium}, manifested
in the properties of thermal fluctuations. This statement may already be understood from the
fact that typical relaxation times of Newtonian solvents ($\sim 10^{-14} s$) are much
smaller than those of Brownian particles ($\sim 10^{-9} s$). Even when colloidal particles
are {\it strongly driven} (e.g. swimmers or by external forces), all known experiments agree
with a theoretical description were the surrounding bath is assumed {\it in equilibrium}.

{\bf As a radically new, important and foundational advancement}, we plan to study the
behavior of probe particles in {\it non-equilibrium baths}, realized by {\it visco-elastic
  solvents}. Visco-elastic fluids exhibit both elastic behavior like a solid, and viscous
behavior like a fluid, a typical example being polymer solutions or melts. Importantly, such
solvents may display large relaxation times (up to seconds), and can thus easily be driven
out of thermal equilibrium by a colloidal probe -- with possibly dramatic consequences; a
variety of novel phenomena are expected to arise. Importantly, such studies will reveal the
properties of non-equilibrium baths which are of manifold importance: They will explicitly
allow to test concepts of non-equilibrium thermodynamics (as outlined in the main
description), and are also expected to be of great technological influence: Most fluids or
solvents used in industry (e.g. oil or paints) as well as found in biology (e.g. blood) are
visco-elastic and out of equilibrium.

{\bf Specifically, the experimental system} will use several types of visco-elastic
solvents, e.g. micellar or polymer solutions, thereby enabling detection of universal
features. Micron sized particles will be driven by different means that are experimental
state of the art: External forces include optical, gravitational or electromagnetic
forces. Importantly, the experimental group in Stuttgart is also among the world leaders
concerning (self propelled) microswimmers (see e.g. Ref.~\cite{Kummel13}), which provide a different and important means of
non-equilibrium driving. Designing experimental realization of micro swimmers in visco
elastic baths is a challenging goal unachieved before. These different types of
perturbations are anticipated to provide a very complete characterization of the
non-equilibrium fluctuations of the bath. One important aspect of this system is that the
probe, being on the micron scale (hence sensitive to fluctuations), is still large compared
to the constituents of the bath (sizes below $\sim 100$~nm), so that the desired {\it
  continuum description} of the bath can be expected to prove useful.

{\bf On the theoretical side}, our main goal will be deriving such effective (continuum)
description, i.e., a Langevin equation for the visco-elastic solvent far from equilibrium,
anticipating that it will crucially depend on the {\it non-equilibrium
  fluctuations}. Technically, we will use a microscopic starting point, including relevant
bath and particle degrees of freedom. The effective description for the probe particle is
obtained via integration of bath degrees of freedom, for which several routes will be used,
including the Zwanzig Mori projection formalism, as well as density functional
theory. Notably, in such coarse graining, typically {\it close to equilibrium conditions}
are assumed, clearly incorrect for the proposed studies. Such coarse graining analysis, in
direct comparison to experimental results, will identify the non-equilibrium properties and
behavior of the fluid, thereby providing fundamentally important insights into
non-equilibrium thermodynamics. The theoretical group in Stuttgart is very experienced in
such coarse graining procedures far from equilibrium (see e.g. Ref.~\cite{Aerov14}) and the described analysis will also
greatly benefit from interactions with our collaborators.


\end{wpdescription}

\begin{tasklist}

\begin{task}[title=task1,id=brown-t1,PM=24,lead=USTUTT,wphases=0-24!1.0]
{\bf Experimental characterization of (universal) aspects of viscoelastic baths, by using different viscoelastic solvents.} This will be achieved by driving particles with different external forces. As a function of driving, we will measure mean (frictional) forces, translational fluctuations (diffusivities), rotational fluctuations and non-equilibrium relaxation times.\\
{\bf Theoretical identification of non-equilibrium signatures of the bath.} Direct comparison of non-equilibrium coarse graining procedures to experimental results will determine the properties of theoretically emerging terms. Support via numerical simulations (Padova) and effective medium theory (Leuven) will be crucial, and comparison to particle based approaches (Leipzig) will be beneficial.\\
%Investigation of possibility of application of nonequilbrium thermodynamcis to such systems, for example by extracting dynamical activities.
\end{task}

\begin{task}[title=task2,id=brown-t2,PM=24,lead=USTUTT,wphases=24-48!1.0]
{\bf Experimental realization of (Brownian) swimmers in the above mentioned viscoelastic media}, an ambitious goal that will use the mechanism for propulsion developped in Stuttgart, based on the presence of a critical (demixing) point. This is an important advancement by itself that allows study of realistic environments of swimmers.\\
Theoretical investigation of nonequilbrium thermodynamcis for non-equilibrium baths, by extracting dynamical activities and nonlinear response functions. Comparison of conclusions to those of other work packages.
\end{task}

\end{tasklist}

\begin{wpdelivs}
  \begin{wpdeliv}[due=24,id=brown-d1,dissem=PU,nature=DEM,lead=USTUTT]
      {Experimental characterization of (universal) properties of non-equilibrium, viscoelastic baths}
  \end{wpdeliv}
  \begin{wpdeliv}[due=24,id=brown-d2,dissem=PU,nature=DEM,lead=USTUTT]
      {Theoretical description of non-equilibrium, viscoelastic baths based on microscopic starting points}
\end{wpdeliv}
  \begin{wpdeliv}[due=48,id=brown-d3,dissem=PU,nature=DEM,lead=USTUTT]
      {Experimental realization of swimmers in natural (viscoelastic) baths}
\end{wpdeliv}
\begin{wpdeliv}[due=48,id=brown-d4,dissem=PU,nature=DEM,lead=USTUTT]
      {Theoretical analysis of non-equilibrium thermodynamics with regards to non-equilibrium baths}
\end{wpdeliv}
\end{wpdelivs}

\end{workpackage
