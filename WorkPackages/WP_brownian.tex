\begin{workpackage}[id=WPbrown,wphases=0-48,
  short=Brown. particles, %XXX act. WP,% for Figure 5.
  title=Brownian particles in nonequilibrium baths, % XXX actual Work Package,
  lead=USTUTT,
  USTUTTRM=96,KULRM=6,ULEIRM=6,UNIPDRM=6]

\newrefsection

\begin{wpobjectives}
\begin{compactitem}
\item Characterization of nonequilibrium baths by driving probe particles in viscoelastic fluids
\item Experimental realization of self propulsion in natural (i.e. viscoelastic) media
  \end{compactitem}
\end{wpobjectives}

\begin{wpdescription}


Despite the success of stochastic control via Langevin dynamics \cite{blickle2006, blickle2007, blickle2012}, the assumption of a thermally equilibrated bath is not fulfilled in most industrially (oil, polymer melts, dense colloidal or micellar suspensions) or biologically relevant (blood, mucus, DNA-solutions) liquids which display visco-elastic properties, {\it i.e.}, elastic behavior like a solid and viscous
behavior like a fluid. As a consequence, such
fluids exhibit large relaxation times (up to seconds), and can thus easily be driven
out of thermal equilibrium by a forced colloidal probe. Within this proposal, we want to study how the local perturbation of visco-elastic systems affects the dynamics and effective interactions of colloidal particles in such media with particular view on e.g. memory effects, many body interactions and local non-linear rheological properties. Such studies will in general reveal the
properties of nonequilibrium baths which are of manifold importance. More explicitly, this approach will provide a rigorous test of the concepts of nonequilibrium thermodynamics (as outlined in the main
description).  It will be a crucial ingredient for the control and exploitation of processes where such nonlinear fluids are used. 


{\bf Specifically, in experiments} will study the dynamical behavior of {\bf externally driven} colloidal particles in different types of visco-elastic
fluids, e.g. micellar or polymer solutions. External forces will be created by optical, magnetic and gravitational forces where considerable expertise is present in the Stuttgart group. In addition to single driven particles, also complex, dynamical forces can be created by acousto-optical deflectors where controlled driving forces can be applied up to several hundred particles. In addition, the use of microfluidic devices, will allow us, to generate well-defined global flow fields \cite{scholz2012}, which will affect the visco-elastic properties of the liquid.

Furthermore, we will also study the motion of {bf self-propelled particles} in visco-elastic media. This will be achieved by a light-induced local demixing process which has been recently demonstrated by the Stuttgart group \cite{kuemmel2013, buttinoni2013, tenHagen2014} and which can be extended to visco-elastic systems. In contrast to externally driven particles, the force-field of self-propelled particles are fundamentally different. Accordingly, qualitative changes in the response of the visco-elastic liquids to self-propelled particles are expected. The comparision of the liquid's response to externally and self-driven particles will provide a more complete characterization of the
nonequilibrium fluctuations of a visco-elastic bath. One important aspect of this system is that the
probe, being on the micron scale (hence sensitive to fluctuations), is still large compared
to the constituents of the bath (sizes below $\sim 100$~nm), so that the desired {\it
  continuum description} of the bath can be expected to prove useful.

{\bf On the theoretical side}, our main goal will be deriving effective (continuum)
descriptions, ready for use for the modeling of dynamics of probes in visco-elastic fluids far from equilibrium. The effective description for the probe particle is
obtained via integration of bath degrees of freedom, for which several routes will be used,
including the Zwanzig-Mori projection formalism, as well as density functional
theory. The theoretical group in Stuttgart is very experienced in
such coarse graining procedures \cite{Aerov14} including  far-from-equilibrium physics \cite{Kruger11,Kruger09}.


{\bf In conclusion} we seek to provide practical guidelines and theoretical studies for controlling probes in nonlinear fluids and to characterize the latter via such probes.



\printbibliography[heading=proposal-bib,env=proposal-env]

\end{wpdescription}

\begin{tasklist}

\begin{task}[title=Experimental setup,id=brown-t1,PM=24,lead=USTUTT,wphases=0-24!0.5]
Implementation of an optical tweezers (including. acousto-optical deflectors), magnetic fields and a temperature- and flow-controlled sample cell in an optical microscope setup. Characterization of the visco-elastic properties of different systems (micellar, DNA solutions, polymers) with conventional rheometry and colloidal-probe rheology. 
\end{task}


\begin{task}[title=Externally driven particles in visco-elastic baths,id=brown-t2,PM=24,lead=USTUTT,wphases=0-24!0.5]
Study the response of externally driven colloids in visco-elastic liquids as a function of the driving amplitude, particle density, the presence of additional flows. As a function of driving, we will
measure mean (frictional) forces, translational fluctuations (diffusivities), rotational
fluctuations and nonequilibrium relaxation times. Application of more complex perturbation fields to the visco-elastic liquid by using magnetic, gravitational and scanned light fields (the latter achieved by acousto-optical modulators).
\end{task}

\begin{task}[title=Theoretical identification of nonequilibrium signatures of the bath,id=brown-t3,PM=24,lead=USTUTT,wphases=0-24!1.0,partners={KUL,UNIPD,ULEI}]
Direct comparison of nonequilibrium coarse graining procedures to experimental results will
determine the properties of theoretically emerging terms. Support via numerical simulations
(Padova) and effective medium theory (Leuven) will be crucial, and comparison to particle
based approaches (Leipzig) will be beneficial.
\end{task}

\begin{task}[title=Self-propelled particles in visco-elastic baths,id=brown-t4,PM=24,lead=USTUTT,wphases=24-48!0.5]
Experimental realization of critical mixtures with visco-elastic properties to achieve light-controlled active Brownian motion in such media. Measurement of single swimmer's trajectories in visco-elastic media and comparison with that of Newtonian systems. Study of gravitactic motion of spherical and non-spherical active particles in visco-elastic systems. 
\end{task}
\begin{task}[title=Nonequilbrium thermodynamics,id=brown-t5,PM=24,lead=USTUTT,wphases=24-48!1.0,partners={KUL,UNIPD,ULEI}]
Theoretical investigation of nonequilibrium thermodynamics for nonequilibrium baths, by
extracting dynamical activities and nonlinear response functions. Comparison of conclusions
to those of other work packages.
\end{task}

\begin{task}[title=Collective behavior of self-propelled particles in visco-elastic baths,id=brown-t6,PM=24,lead=USTUTT,wphases=24-48!0.5]
Investigation of particle interactions mediated by transient flow and strain fields, created by active particles. Consequences of such interactions for collective phenomena, like the formation of clusters. Structure formation in mixtures of active and passive particles in visco-elastic media.  
\end{task}


\end{tasklist}

\begin{wpdelivs}
  \begin{wpdeliv}[due=24,id=brown-d1,dissem=PU,nature=DEM,lead=USTUTT,miles=data1]
      {Experimental set-up for driven systems in visco-elastic media, characterization of (universal) properties of viscoelastic media}
  \end{wpdeliv}
  \begin{wpdeliv}[due=24,id=brown-d2,dissem=PU,nature=DEM,lead=USTUTT,miles=data1]
      {Theoretical description of nonequilibrium, visco-elastic baths based on microscopic starting points}
\end{wpdeliv}
 
  \begin{wpdeliv}[due=48,id=brown-d3,dissem=PU,nature=DEM,lead=USTUTT,miles=final]
       {Measurements of the response of self- driven particles in visco-elastic media}
 \end{wpdeliv}
\begin{wpdeliv}[due=48,id=brown-d4,dissem=PU,nature=DEM,lead=USTUTT,miles=final]
      {Theoretical analysis of nonequilibrium thermodynamics with regards to nonequilibrium baths}
\end{wpdeliv}
 \begin{wpdeliv}[due=48,id=brown-d5,dissem=PU,nature=DEM,lead=USTUTT,miles=final]
      {Measurements of the collective phenomena of driven particles and mixtures with passive particles in visco-elastic media}
\end{wpdeliv}
\end{wpdelivs}

\end{workpackage}
