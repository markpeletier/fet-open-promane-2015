\begin{workpackage}[id=WPcompress,wphases=0-48,
  short=Nonequilibrium compressibility, %XXX act. WP,% for Figure 5.
  title=Nonequilibrium compressibility, % XXX actual Work Package,
  lead=UNIPD,
  UNIPDRM=72,
  KULRM=12]

\newrefsection

\begin{wpobjectives}
  \begin{compactitem}
  \item Exploit the negative response to compression of systems holding a heat flow for the design of meta-materials.
  \item Apply effective fluctuation-response relations for the upscaling of industrial processes.
  \item Exhibit novel architectures of polymers and membranes under flowing conditions.
  \end{compactitem}
\end{wpobjectives}

\begin{wpdescription}

An important class of systems for which we need to understand the nonequilibrium response is that of solids 
in a temperature gradient. Examples include strongly
heated micro-cantilevers \cite{AGBB15} and high-quality table-top oscillators of gravitational wave detectors \cite{Cet13}.
In fact, compressibility of solids in a temperature gradient is a quite unexplored subject.
This WP plans to fill this gap, by understanding how the response to compression takes place and how it
may be predicted by just knowing suitable (unperturbed) steady-state correlations.
That knowledge is important for a second, main stage of this WP, as explained next.


The possibility to extract energy from an installed heat channel is the mechanism that we want to
analyze, to see if it can lead to the behaviour of negative differential compressibility,
i.e.~the system expands when compressed. 
Nonequilibrium response may be written as an entropic term minus a frenetic one \cite{BMW09}.
In our case the former is the correlation between the size of the system and the entropy produced by the compression.
The frenetic term is another correlation between size and dynamical aspects that can be monitored numerically.
This scheme thus furnishes us with a guideline to assess whether simulations of a specific material are yielding data 
going in the direction of negative compressibility. In particular, since typically the entropic term is positive, we
are encouraged to privilege the development of models that show a large frenetic term, so that its subtraction from
the entropic one gives a total negative response.
We have in this way delineated a guided trial-and-error strategy, in which the notions of 
nonequilibrium statistical mechanics help us in faster developments of numerical studies. This is expected to
speed up the research if compared to a blind trial-and-error strategy.


We can consider this WP as a high reward research plan. Indeed, relevant technological applications
may be boosted by a successful achievement in finding models of materials that have negative compressibility.
Metamaterials are nowadays much studied \cite{NM12,CG15} because, for example, artificial muscles,
next-generation pressure sensors, and micro-actuators would be
more easily realized by using materials with a negative compressibility. We will investigate
if steady nonequilibrium conditions could produce this effect, as opposed to the recently considered 
mechanism of rearrangements from metastable states \cite{NM12} or of enhanced expansions in orthogonal directions
due to a wine-rack structure of the materials \cite{CG15}. Thus, the anomalous response we seek is in the direction
of the compression and it is due to compression-enhanced extraction of energy from a heat channel,
which is prompt, excites the system expanding it, 
and does not require going through an hysteresis of the material. 

\printbibliography[heading=proposal-bib,env=proposal-env]

\end{wpdescription}

\begin{tasklist}

  \begin{task}[title=TASK1,id=task1,PM=3,lead=UNIPD,wphases={0-12!1,12-24!0.5}]
 
    Characterizing the response to compression
    for models of solids experiencing a heat flow due to different boundary temperatures.
    By applying available theoretical formulations to the study of numerical simulations,
    the main aim is to make a setup for the other tasks.
    
  \end{task}

  \begin{task}[title=TASK2,id=task2,PM=3,lead=UNIPD,partners=KUL,wphases={12-24!0.5,24-36!1}]

    From simulations and {\it ab initio} numerical integration, finding what microscopic details
    are needed  for
    predicting the response of the system's length to an increased compression.
    The main task of this WP is to
        find peculiar inter-particle potentials that may lead to negative 
        compressibility in solids experiencing heat flows.     
  \end{task}

  \begin{task}[title=TASK3,id=task3,PM=6,lead=UNIPD,wphases={18-24!0.5,24-48!1},partners={KUL,ULEI}]
Important probes for many industrial and medical applications are polymers, often found in nonequilibrium and/or nonlinear environments.
We set as task to give the typical shape and the architecture of these polymers, as chemical reactivity can much depend on that.
That will lead to improved control of these features via nonequilibrium monitoring.
It is important to reach guidelines for enhanced or selective reactivity, depending on the case, but for sure the topics of the workpackage and of the project in general are extremely important for understanding them.
There will collaborations here with the groups in Leuven (also with Enrico Carlon) and with the Leipzig node.  Regular visits and exchange lectures between Leuven and Padova are already in place.
  \end{task}

\end{tasklist}

\begin{wpdelivs}
  \begin{wpdeliv}[due=24,id=mydeliv1,dissem=PU,nature=DEM,lead=UNIPD]
      {First deliverable, after 2 years: characterization of compressibility for solids between two temperatures.}
  \end{wpdeliv}
  \begin{wpdeliv}[due=36,id=mydeliv2,dissem=PU,nature=DEM,lead=UNIPD]
      {Second deliverable, after 3 years: clear picture on how microscopic details are needed to predict compressibility from experiments}
  \end{wpdeliv}
  \begin{wpdeliv}[due=48,id=mydeliv3,dissem=PU,nature=DEM,lead=UNIPD]
      {Third deliverable, after 4 years: characterization of the compressibility for a wide class of heat-conduction models, understanding of whether negative compressibility is achievable in real systems.}
\end{wpdeliv}
\end{wpdelivs}




\end{workpackage}
