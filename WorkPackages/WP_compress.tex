\begin{workpackage}[id=WPcompress,wphases=0-48,
  short=Nonequilibrium compressibility, %XXX act. WP,% for Figure 5.
  title=Nonequilibrium compressibility, % XXX actual Work Package,
  lead=UNIPD,
  UNIPDRM=72,
  KULRM=12]

\newrefsection

\begin{mdframed}
\mobjectives
%
  \begin{compactitem}
  \item Design meta-materials with a negative response to compression when holding a heat flow.
  \item Apply effective fluctuation-response relations for the upscaling of industrial processes.
  \item Exhibit novel architectures of polymers and membranes under flowing conditions.
  \end{compactitem}

\mdescription
%
Strongly heated micro-cantilevers \cite{AGBB15} and high-quality table-top oscillators of
gravitational wave detectors \cite{Cet13} are examples of systems in a temperature gradient
where the nonequilibrium response is of practical importance, yet remains unexplored.
%
We will characterize this response and design a predictive theory that relies on unperturbed
steady-state correlations.

The possibility to extract energy from an installed heat channel is the mechanism that we want to
analyze, to see if it can lead to the behaviour of negative differential compressibility,
i.e.~the system expands when compressed. 
Nonequilibrium response may be written as an entropic term minus a frenetic one \cite{BMW09}.
In our case the former is the correlation between the size of the system and the entropy produced by the compression.
The frenetic term is another correlation between size and dynamical aspects that can be monitored numerically.
%
These two contributions allow an operational assessment of negative compressibility and will
serve as a guide for a trial-and-error strategy, using nonequilibrium physics to accelerate the
exploration of new materials.

Metamaterials \cite{NM12,CG15} are candidates for next-generation pressure sensors and
micro-actuators. Negative compressibility has the potential for enabling these technologies
and even more ambitious applications such as artificial muscles.
%
Steady nonequilibrium conditions are a new ingredient to produce this effect, as opposed to the recently considered
mechanism of rearrangements from metastable states \cite{NM12} or of enhanced expansions in orthogonal directions
due to a wine-rack structure of the materials \cite{CG15}.
%
The mechanism that we propose is due to compression-enhanced extraction of energy from a
heat channel. It is fast and does not require going through an hysteresis of the material.

\printbibliography[heading=proposal-bib,env=proposal-env]

\end{mdframed}

\begin{tasklist}

  \begin{task}[title=TASK1,id=task1,PM=3,lead=UNIPD,wphases={0-12!1,12-24!0.5}]
  We characterize the response to compression
  for models of solids experiencing a heat flow due to different boundary temperatures.
  By applying available theoretical formulations to the study of numerical simulations,
  we prepare appropriate models for subsequent tasks.
  \end{task}

  \begin{task}[title=TASK2,id=task2,PM=3,lead=UNIPD,partners=KUL,wphases={12-24!0.5,24-36!1}]
  Identify the microscopic parameters needed to predict the response
  of a system's length under increased compression, using simulations
  and {\it ab initio} numerical integration.
  %
  The main task of this WP is to find peculiar inter-particle
  potentials that may lead to negative compressibility in solids
  experiencing heat flows.
  \end{task}

  \begin{task}[title=TASK3,id=task3,PM=6,lead=UNIPD,wphases={18-24!0.5,24-48!1},partners={KUL,ULEI}]
  Characterize the typical shape and the architecture of polymers used
  as probes of nonequilibrium and develop a corresponding control
  strategy.
%
  On this basis, we will also quantify the impact on enhanced or
  selective reactivity of those polymers with surrounding chemicals.
  There will collaborations here with the groups in Leuven (with whom
  frequent visits already occur) and in Leipzig.
  \end{task}

\end{tasklist}

\begin{wpdelivs}
  \begin{wpdeliv}[due=24,id=mydeliv1,dissem=PU,nature=R,lead=UNIPD]
  {Characterization the compressibility for solids between two temperatures.}
  \end{wpdeliv}
  \begin{wpdeliv}[due=36,id=mydeliv2,dissem=PU,nature=DEM,lead=UNIPD]
  {Identification of the microscopic parameters needed to predict compressibility.}
  \end{wpdeliv}
  \begin{wpdeliv}[due=48,id=mydeliv3,dissem=PU,nature=DEM,lead=UNIPD]
  {Characterization of the compressibility for a wider class of heat-conducting materials, and
    application of negative compressibility to realistic models.}
\end{wpdeliv}
\end{wpdelivs}




\end{workpackage}
