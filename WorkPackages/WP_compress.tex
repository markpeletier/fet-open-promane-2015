\begin{workpackage}[id=WPcompress,wphases=0-48,
  short=Nonequilibrium compressibility, %XXX act. WP,% for Figure 5.
  title=Nonequilibrium compressibility, % XXX actual Work Package,
  lead=UNIPD,
  UNIPDRM=72,
  KULRM=12]

\newrefsection

\begin{wpobjectives}
  \begin{compactitem}
  \item Design meta-materials with a negative response to compression when holding a heat flow.
  \item Apply effective fluctuation-response relations for the upscaling of industrial processes.
  \item Exhibit novel architectures of polymers and membranes under flowing conditions.
  \end{compactitem}
\end{wpobjectives}

\begin{wpdescription}

An important class of systems for which we need to understand the nonequilibrium response is that of solids 
in a temperature gradient. Examples include strongly
heated micro-cantilevers \cite{AGBB15} and high-quality table-top oscillators of gravitational wave detectors \cite{Cet13}.
In fact, compressibility of solids in a temperature gradient is a quite unexplored subject.
This WP plans to fill this gap, by understanding how the response to compression takes place and how it
may be predicted by just knowing suitable (unperturbed) steady-state correlations.
That knowledge is important for a second, main stage of this WP, as explained next.


The possibility to extract energy from an installed heat channel is the mechanism that we want to
analyze, to see if it can lead to the behaviour of negative differential compressibility,
i.e.~the system expands when compressed. 
Nonequilibrium response may be written as an entropic term minus a frenetic one \cite{BMW09}.
In our case the former is the correlation between the size of the system and the entropy produced by the compression.
The frenetic term is another correlation between size and dynamical aspects that can be monitored numerically.
This scheme thus furnishes us with a guideline to assess whether simulations of a specific material are yielding data 
going in the direction of negative compressibility. In particular, since typically the entropic term is positive, we
are encouraged to privilege the development of models that show a large frenetic term, so that its subtraction from
the entropic one gives a total negative response.
We have in this way delineated a guided trial-and-error strategy, in which the notions of 
nonequilibrium statistical mechanics help us in faster developments of numerical studies. This is expected to
speed up the research if compared to a blind trial-and-error strategy.


We can consider this WP as a high reward research plan. Indeed, relevant technological applications
may be boosted by a successful achievement in finding models of materials that have negative compressibility.
Metamaterials are nowadays much studied \cite{NM12,CG15} because, for example, artificial muscles,
next-generation pressure sensors, and micro-actuators would be
more easily realized by using materials with a negative compressibility. We will investigate
if steady nonequilibrium conditions could produce this effect, as opposed to the recently considered 
mechanism of rearrangements from metastable states \cite{NM12} or of enhanced expansions in orthogonal directions
due to a wine-rack structure of the materials \cite{CG15}. Thus, the anomalous response we seek is in the direction
of the compression and it is due to compression-enhanced extraction of energy from a heat channel,
which is prompt, excites the system expanding it, 
and does not require going through an hysteresis of the material. 

\printbibliography[heading=proposal-bib,env=proposal-env]

\end{wpdescription}

\begin{tasklist}

  \begin{task}[title=TASK1,id=task1,PM=3,lead=UNIPD,wphases={0-12!1,12-24!0.5}]
  We characterize the response to compression
  for models of solids experiencing a heat flow due to different boundary temperatures.
  By applying available theoretical formulations to the study of numerical simulations,
  we prepare appropriate models for subsequent tasks.
  \end{task}

  \begin{task}[title=TASK2,id=task2,PM=3,lead=UNIPD,partners=KUL,wphases={12-24!0.5,24-36!1}]
  Identify the microscopic parameters needed to predict the response
  of a system's length under increased compression, using simulations
  and {\it ab initio} numerical integration.
  %
  The main task of this WP is to find peculiar inter-particle
  potentials that may lead to negative compressibility in solids
  experiencing heat flows.
  \end{task}

  \begin{task}[title=TASK3,id=task3,PM=6,lead=UNIPD,wphases={18-24!0.5,24-48!1},partners={KUL,ULEI}]
  Characterize the typical shape and the architecture of polymers used
  as probes of nonequilibrium and develop a corresponding control
  strategy.
%
  On this basis, we will also quantify the impact on enhanced or
  selective reactivity of those polymers with surrounding chemicals
  There will collaborations here with the groups in Leuven (with whom
  frequent visits already occur) and in Leipzig.
  \end{task}

\end{tasklist}

\begin{wpdelivs}
  \begin{wpdeliv}[due=24,id=mydeliv1,dissem=PU,nature=R,lead=UNIPD]
  {Characterization the compressibility for solids between two temperatures.}
  \end{wpdeliv}
  \begin{wpdeliv}[due=36,id=mydeliv2,dissem=PU,nature=DEM,lead=UNIPD]
  {Identification of the microscopic parameters needed to predict compressibility.}
  \end{wpdeliv}
  \begin{wpdeliv}[due=48,id=mydeliv3,dissem=PU,nature=DEM,lead=UNIPD]
  {Characterization of the compressibility for a wider class of heat-conducting materials, and
    application of negative compressibility to realistic models.}
\end{wpdeliv}
\end{wpdelivs}




\end{workpackage}
